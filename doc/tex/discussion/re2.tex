\subsection{RE2}

\begin{itemize}

   \item Does not support extended regular expressions, only basic RE.

   \item Automata based approach.

\end{itemize}


The RE2 engine by Russ Cox is a regular expression matcher based on finite
automata. In the first step, it constructs an NFA from the expression. When
matching an input string, it constructs a DFA on the fly and caches the states.
If the DFA cache exceeds a certain size, it is cleared and the DFA construction
starts over.

RE2 implements several low-level optimisations and exhibits high performance. It
provides specialised functions for specific matching tasks, e.g. a function that
only solves the word problem, that is, it tests whether the complete input is a
word in the language described by the regular expression. Another function may
test whether any substring of the input can be matched by the expression.

In addition, it performs several analyses over the regular expression that allow
it to choose specialised paths and algorithms for common cases of simple regular
expressions. For example, if the engine can always know which sub-match a
character is part of, it can avoid copying sub-match boundary sets. For example
in $(x:a^*)(y:b)$, all $a$s are assigned to the match variable $x$, and the
final $b$ is assigned to $y$, whereas in $(x:a^*)(y:a)$, after reading $aa$, it
can not yet be decided which variable the letters will be assigned to, resulting
in a boundary set with two elements: $\{\{(x,aa)\},\{(x,a),(y,a)\}\}$.

The problem solved by our pattern sub-matching with partial derivatives is the
most difficult one, and RE2 solves it by first running the DFA on the input to
find the end of the matched substring. After that, it runs another DFA in which
all concatenations of regular expressions are reversed, in order to find the
beginning of the matched substring. Finally, it runs an NFA to locate the
sub-matches.


\subsubsection{Automata construction}

The time taken by RE2 to compile a regular expression into an automaton is,
according to \cite{regexp3}, about three to four times more than by PCRE. The
resulting automaton object can become large, because of the cached DFA, but RE2
limits the size to a user-defined limit. By constructing DFA states on demand
rather than constructing the complete DFA up front, RE2 avoids the exponential
complexity of a full NFA state subset construction.


\subsubsection{Matching performance}

Using an automata based approach, RE2 avoids exponential time complexity for
regular expression matching, which a backtracking algorithm exposes for example
when matching $a^n$ with the expression $a?^na^n$. Russ Cox explains this
runtime behaviour in more detail in \cite{regexp1}. Common case optimisations
help RE2 outperform PCRE on large inputs, and the engine is not much slower than
PCRE on small strings.


\subsubsection{Accepted syntax}

RE2 accepts most common regular expression operators, but does not in any way
support negation or intersection of subexpressions. Back-references are not
supported in any automata based engine, and although RE2 contains a backtracking
implementation, it does not expose this in its API or syntax.


% vim:tw=80
