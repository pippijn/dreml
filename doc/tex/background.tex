\chapter{Background}
\label{background}

\section{Derivatives of regular expression}

Janusz Brzozowski introduces a method of symbolically computing the left
derivative of a regular expression over an input symbol $l$. After constructing
a deterministic finite automaton for the language, one can follow the transition
on $l$ and take the target state as the start state of a new, derived automaton.
The language this automaton accepts is the same as the language accepted by an
automaton constructed from a Brzozowski derivative over $l$.

Constructing a DFA from a regular expression by the method described in
\cite{thompson} generally goes via an epsilon NFA\footnote{Nondeterministic
Finite Automaton} by computing an epsilon closure and then performing a subset
construction of NFA states into DFA states. On the contrary, Brzozowski's
derivative method directly results in a DFA.

The derivative operation is defined on basic regular expressions $r$ with an
input alphabet $\Sigma$.

\begin{defn}[Basic regular expression syntax]
   \label{defn-bre}
   Abstract syntax:

   \begin{tabular}{lrll}
      $r$	& $::=$	& $r+r$				& Choice		\\
		& $|$	& $(r,r)$			& Concatenation		\\
		& $|$	& $r^*$				& Kleene star		\\
		& $|$	& $\varepsilon$			& Empty word		\\
		& $|$	& $\phi$			& Empty language	\\
		& $|$	& $l \in Sigma$			& Letters		\\
   \end{tabular}
\end{defn}

On this syntax, Brzozowski defines the left derivative.

\begin{defn}[Brzozowski's regular expression derivative]
   \label{defn-deriv}
   Abstract syntax:

   \begin{tabular}{lrl}
      $\phi \setminus l$	& $=$	& $phi$	\\
      $\varepsilon \setminus l$	& $=$	& $phi$	\\
      $l_1 \setminus l_2$	& $=$	&
          $\begin{cases}
             \varepsilon & \text{if } l_1 = l_2 \\
             \phi & \text{otherwise}
          \end{cases}$	\\
      $(r_1 + r_2) \setminus l$	& $=$	& $r_1 \setminus l + r_2 \setminus l$	\\
      $(r_1 + r_2) \setminus l$	& $=$	&
          $\begin{cases}
             (r_1 \setminus l, r_2) & \text{if } \varepsilon \not\in L(r_1\esetsOfRe) \\
             (r_1 \setminus l, r_2) + r_2 \setminus l & \text{otherwise}
          \end{cases}$	\\
      $r^* \setminus l$		& $=$	& $(r \setminus l, r^*)$	\\
   \end{tabular}
\end{defn}

An automaton is produced by iterating over the set of input symbols (the
alphabet $\Sigma$) and producing a derivative for each symbol. The resulting
derivated terms are then subject to the same process. This is repeated until no
new derivatives appear. As the automaton produced is in fact a deterministic
finite automaton, the number of states produced by this method is exponential in
the expression length. This is problematic for certain expressions such as
\verb!(a?){n}(a){n}!. Although the runtime of a derivative automaton is linear
in the input length and has a small constant factor, the automaton construction
can become a lengthy operation.


\section{Partial derivatives}
\label{pd}

In order to reduce the time spent in constructing the matching automaton,
Valentin M. Antimirov developed the concept of partial derivatives of regular
expressions in \cite{antimirov}. Instead of a single derivated term for each
input symbol, partial derivatives may produce an arbitrary number of terms.
Every time the derivative operation is performed for an choice expression, the
possible choices are split as a set of derivated terms.

Antimirov proved that the set of partial derivatives for any given expression is
finite. More importantly, he asserts that the number of partial derivatives of a
regular expression and recursively of all its derivated terms is at most one
more than the number of occurrences of symbols from the input alphabet. In other
words, the number of states in the non-deterministic matching automaton
constructed from partial derivatives grows linearly in the length of the initial
expression.

Consequently, an automaton construction algorithm based on Antimirov's ideas can
run in linear time over the regular expression size, and the actual NFA
simulation still runs in linear time over the input size, if all branches are
investigated simultaneously. The resulting matcher takes less memory, but
slightly more time, since the sets of NFA states the automaton is currently in
must be computed on the fly. As these sets are by themselves DFA states, they
may be cached in order to lazily perform a subset construction, resulting in a
deterministic finite automaton.


\section{Pattern submatching}

Based on the ideas by Brzozowski and Antimirov, Martin Sulzmann and Kenny Zhuo
Ming Lu introduced the notion of regular expression submatching to partial
derivatives. In addition to solving the word problem, that is to determine
whether a regular expression matches the input, this advancement allows to
assign parts of the matched sequence to different variables. By inspecting the
resulting matching environment, one can extract the subsequences of interest.
This is particularly useful when dealing with programming languages or search
and replace functionality in editors.

\subsection{Syntax}

In \cite{pdpat}, Sulzmann and Lu define the abstract syntax for expression
patterns:

\begin{defn}[Regular expression patterns]
   \label{defn-pat}
   Syntax:

   \begin{tabular}{lrll}
      $p$	& $::=$	& $(x:r)$	& Variables Base	\\
		& $|$	& $(x:p)$	& Variables Group	\\
		& $|$	& $p+p$		& Choice		\\
		& $|$	& $(p,p)$	& Concatenation		\\
		& $|$	& $p^*$		& Kleene star		\\
   \end{tabular}
\end{defn}

A variables base contains a regular expression as defined above and does not
contain any more patterns, so that the derivative operation on this class of
patterns can be defined in terms of definition \ref{defn-deriv}. The definition
and precise description of pattern derivatives is not repeated here, but can be
found in \cite{pdpat}.

The definition of pattern partial derivatives is such that it yields a set of
derivated terms along with a transition or matching function. In the first step
of the implementation, the match was recorded in the pattern structure itself,
building on the following syntax:

\begin{defn}[Regular expression patterns with match]
   \label{defn-pat-with-match}
   Syntax:

   \begin{tabular}{lrll}
      $p$	& $::=$	& $(x|w:r)$	& Variables Base  with match $m$	\\
		& $|$	& $(x|w:p)$	& Variables Group with match $m$	\\
		& $|$	& $\dots$	& The rest is the same as above.	\\
   \end{tabular}
\end{defn}

An important additional requirement to the partial derivatives is that the
operation should no longer yield an unordered set, but a set ordered by the
appearance of their origins in the expression operand. Thus, the partial
derivative of $(x:(a,c))(y:(a,b))$ over the input symbol $a$ must yield
$[(x:c),(y:b)]$ in that order. An unordered set is allowed within expressions
not containing pattern variables, so that $(a,c)(a,b) \setminus_p a$ may yield
$\{(b),(c)\}$.

\subsection{Sub-match extraction}

Programming languages usually contain a finite set of keywords and lexical rules
for different kinds of tokens. OCaml, for instance, specifies that an sequence
of Latin letters, starting with a capital letter, is a module name or a type
constructor name. A sequence starting with a minuscule is a type or variable
name, unless it is also a keyword. More complex rules exist for string literals,
character literals and type variables. OCaml comments can be nested, and
therefore not be matched by a regular expression.

When building a compiler for a programming language, one of the first steps
consists in tokenising or lexing the input. Considering a language containing
the keywords \verb!let! and \verb!in!, as well as the operators \verb!=! and
\verb!+!, we can create a regular expression
%
\[((l,e),t) + (i,n) + \verb'=' + \verb'+'\]
%
which matches a single lexical element of the language. However, without
submatching, we can not know which of the elements it actually is. In
\cite{pdpat}, Sulzmann and Lu describe an efficient method for constructing a
tagged NFA that can answer this question.

With submatching, we can write the above expression as the pattern
%
\[(x_\text{let}:((l,e),t)) + (x_\text{in}:(i,n)) + (x_\text{eq}:\verb'=') +
(x_\text{plus}:\verb'+')\]
%
The submatching automaton will yield a matching environment in which exactly
one of the variables $x_i$ is defined, allowing us to decide what kind of token
was matched and extract the precise spelling of that token. The latter is useful
in the presence of multiple spellings, which exists in the C programming
language in the form of digraphs and trigraphs, as well as in most programming
languages in the form of identifiers.

The expression above does not yet account for such constructs, so for
demonstration purposes, we introduce the \textit{identifier} token. The regular
expression for identifiers in our small lambda language is \verb![a-z]+!,
allowing only lower case Latin letters for simplicity. The lexer pattern now
needs to be extended, resulting in
%
\[(x_\text{let}:((l,e),t)) + (x_\text{in}:(i,n)) + (x_\text{id}:(a+\dots+z)^n) +
\dots\]
%
This expression contains an ambiguity, because the sequence \verb!let! can be
bound to either $x_\text{let}$ or $x_\text{id}$. The solution to this ambiguity
lies in a concept called \textit{matching strategies}, also introduced to
partial derivatives by \cite{pdpat}.

\subsection{Matching strategies}

Matching strategies define the observable submatching semantics on the pattern
syntax. Sulzmann and Lu introduce two possible strategies: Greedy Left-Most and
POSIX matching. Greedy Left-Most is the way Perl and most other common matching
algorithms resolve ambiguities. It is conceptionally simple: the first matching
sub-expression, when reading the pattern from left to right, is selected and the
matched subsequence is assigned to its variable.
% This strategy comes naturally with partial derivatives and non-deterministic
% automata. TODO: Why?
POSIX matching requires the longest match to be recorded in the environment. For
our simple lambda language example, both strategies would yield the same result,
but for example the pattern $((x:A+AB),(y:BAA+AA))$ matching the input $ABAA$
would, for greedy left-most matching, result in the match $\{x:AB, y:AA\}$, but
for POSIX matching it would yield $\{x:A, y:BAA\}$.

Without a matching strategy, the environment would contain every possible match,
which can be exponentially many. A strategy ensures that the algorithm keeps its
linear time complexity in the input length.


\section{Partial derivatives of extended regular expressions}

Caron, Champarnaud and Mignot extend Antimirov's partial derivatives with
additional boolean operators. In their paper \cite{pdere}, they introduce
negation $\neg$ and intersection $\cap$ and define them, as well as the existing
union or choice operator $+$, in terms of operations yielding a disjunctive
normal form of derivated terms.

First, we define the abstract syntax of extended regular expressions.

\begin{defn}[Extended regular expressions]
   \label{defn-ere-syn}
   Syntax:

   \begin{tabular}{lrll}
      $r$	& $::=$	& $r+r$				& Choice		\\
		& $|$	& $(r,r)$			& Concatenation		\\
		& $|$	& $r_e \cap r_e$		& Intersection		\\
        	& $|$	& $\neg r_e$			& Negation		\\
        	& $|$	& $r_e^n$			& Repetition		\\
		& $|$	& $r^*$				& Kleene star		\\
		& $|$	& $\varepsilon$			& Empty word		\\
		& $|$	& $\phi$			& Empty language	\\
		& $|$	& $l \in Sigma$			& Letters		\\
   \end{tabular}
\end{defn}


\subsection{Sets of expression sets}

\cite{pdere} describes a way to break the derivative of a regular expression
into a union of derivated terms. This union is equivalent to the choice operator
$+$ on the resulting terms. The article defines a representation of a partial
derivative as a set of expression sets in which the expression sets represent
the intersection of terms, and the sets of expression sets represent their
union.

\begin{defn}[Sets of expression sets]
   \label{defn-esets}
   Syntax:

   \begin{tabular}{lll}
      $\eset$	& $::=$ & $\{r_1, r_2, \dots, r_n\}$ \\
      $\esets$	& $::=$ & $\{\eset_1, \eset_2, \dots, \eset_n\}$ \\
   \end{tabular}
\end{defn}

It can be recognised that representation of regular expressions is essentially a
disjunctive normal form (DNF) of atomic regular expressions.

\subsubsection{Conversions}

We can define translations from the above extended regular expression syntax to
the sets of sets representation. Let $\esets$ be a set of expression sets,
$\eset$ an expression set, and $r$ an expression, then we can define the
following operators:

\begin{defn}[Regular expression to expression sets]
   \label{defn-re2esets}
   Conversions from regular expressions.

   Expression set from regular expression:
   $\cdot\esetOfRe :: r \to \eset$

   \begin{tabular}{lll}
      $(r_1 \cap r_2)\esetOfRe$	& $=$	& $r_1\esetOfRe \cup r_2\esetOfRe$	\\
      $r\esetOfRe$			& $=$	& $\{r\}$			\\
   \end{tabular}

   Set of expression sets from regular expression:
   $\cdot\esetsOfRe :: r \to \esets$

   \begin{tabular}{lll}
      $(r_1 + r_2)\esetsOfRe$	& $=$	& $r_1\esetsOfRe \cup r_2\esetsOfRe$	\\
      $r\esetsOfRe$		& $=$	& $\{r\esetOfRe\}$			\\
   \end{tabular}
\end{defn}

The following example shows how a regular expression is transformed into the DNF
representation of atomic expressions.

\needspace{2cm}
\begin{eg}
   $((aa \cap a^*) + b)\esetsOfRe$

   \begin{tabular}{lll}
      $((aa \cap a^*) + b)\esetsOfRe$
      & $=$	& $((aa \cap a^*)\esetsOfRe) \cup (b\esetsOfRe)$		\\
      & $=$	& $\{(aa \cap a^*)\esetOfRe\} \cup \{b\esetsOfRe\}$		\\
      & $=$	& $\{(aa)\esetOfRe \cup (a^*)\esetOfRe\} \cup \{\{b\}\}$	\\
      & $=$	& $\{\{aa\} \cup \{a^*\}\} \cup \{\{b\}\}$			\\
      & $=$	& $\{\{aa, a^*\}, \{b\}\}$					\\
   \end{tabular}
\end{eg}

If the operator at the highest level of the syntax tree is not the choice
operator $+$, the resulting set will contain exactly one expression set. If the
expression contains no intersections, the expression set will contain exactly
one atomic expression. A negation at the highest level is an example of such a
case: $\neg((aa \cap a^*) + b)\esetsOfRe = \{\{\neg((aa \cap a^*) + b)\}\}$

The inverse operation, translating a set of expression sets back to a regular
expression, can be defined analogously.

\begin{defn}[Expression sets to regular expressions]
   \label{defn-esets2re}
   Conversions from expression sets back to regular expressions.

   Regular expression from expression set: $\cdot\reOfEset :: \eset \to r$

   \begin{tabular}{lll}
      $\{r\}\reOfEset$			& $=$	& $r$				\\
      $(\{r\} \cup \eset)\reOfEset$	& $=$	& $r \cap \eset\reOfEset$	\\
   \end{tabular}

   Regular expression from a set of expression sets: $\cdot\reOfEsets :: \esets \to r$

   \begin{tabular}{lll}
      $\{\eset\}\reOfEsets$		& $=$	& $\eset\reOfEset$		\\
      $(\{\eset\} \cup \esets)\reOfEsets$	& $=$	& $r + \esets\reOfEsets$	\\
   \end{tabular}
\end{defn}

The operation is not defined on empty sets, thus the set of expression sets
representing the empty language expression must be the set containing the set
with the empty language symbol: $\{\{\phi\}\}$.

\subsubsection{Operators on sets of expression sets}

Caron et al. define several operators on sets of expression sets and use them to
define the partial derivative of an extended regular expression\cite{pdere}. We
define these operators in the syntax used in \cite{pdpat}.

\begin{defn}[Distributive laws on sets of expression sets]
   \label{defn-eset-ops}
   The normal distributive laws can be extended to sets of expression sets.

   \begin{tabular}{lll}
      $\esets \circledcdot r_2$
         & $=$
         & $\{ \{ (r_1, r_2) \where r_1 \in \eset \} \where \eset \in \esets \}$
         \\

      $\esets_1 \circledcap \esets_2$
         & $=$
         & $\{
              \eset_1 \cup \eset_2
              \where \eset_1 \in \esets_1, \eset_2 \in \esets_2
           \}$
         \\

      $\circledneg\esets$
         & $=$
         & $\{ \{ \neg r \where r \in \eset \} \where \eset \in \esets \}$
         \\
   \end{tabular}
\end{defn}

We can recognise that these operators are essentially distributive laws defined
on sets of expression sets. E.g. the $\circledcdot$ operator concatenates an
expression to every expression in every set in the expression set $\esets$, the
$\circledneg$ operator negates every expression in the set.

The following examples use the extended regular expression syntax as specified
in definition \ref{defn-syn}.

\needspace{2cm}
\begin{eg}
   $((a^* \cap aa) + ab)\esetsOfRe \circledcdot c^*$

   \begin{tabular}{lll}
      $=$ &
         $\{\{a^*, aa\}, \{ab\}\} \circledcdot c^*$
      \\ $=$ &
         $\{ \{ (r_1, c^*) \where r_1 \in \eset \} \where \eset \in \{\{a^*, aa\}, \{ab\}\} \}$
      \\ $=$ &
         $\{ \{ (r_1, c^*) \where r_1 \in \{a^*, aa\} \}, \{ (r_1, c^*) \where r_1 \in \{ab\} \} \}$
      \\ $=$ &
         $\{ \{ (a^*, c^*), (aa, c^*) \}, \{ (ab, c^*) \} \}$
   \end{tabular}

   The resulting set of expression sets can be converted back to a regular
   expression by the $\cdot\reOfEsets$ operator, yielding $(a^*c^* \cap aac^*) +
   abc^*$.
\end{eg}

\begin{eg}
   $((a^* \cap aa) + b^*)\esetsOfRe \circledcap (a^* + (b^* \cap bb))\esetsOfRe$

   \begin{tabular}{lll}
      $=$ &
         $\{\{a^*, aa\}, \{b^*\}\} \circledcap \{\{a^*\}, \{b^*, bb\}\}$
      \\ $=$ &
         $\{
            \eset_1 \cup \eset_2
            \where \eset_1 \in \{\{a^*, aa\}, \{b^*\}\}
            , \eset_2 \in \{\{a^*\}, \{b^*, bb\}\}
         \}$
      \\ $=$ &
         $\{
            \{a^*, aa\} \cup \{a^*\},
            \{a^*, aa\} \cup \{b^*, bb\},
            \{b^*\} \cup \{a^*\},
            \{b^*\} \cup \{b^*, bb\}
          \}$
      \\ $=$ &
         $\{
            \{a^*, aa\},
            \{a^*, aa, b^*, bb\},
            \{b^*, a^*\},
            \{b^*, bb\}
          \}$
   \end{tabular}

   Applying operator $\reOfEsets$ on the result yields the regular expression
   $(a^* \cap aa) + (a^* \cap aa \cap b^* \cap bb) + (b^* \cap a^*) + (b^* \cap
   bb)$. In this case, we could simplify the expression, using the knowledge
   that $L(a^* \cap b^*) = \emptyset$, and obtain the final result $(a^* \cap
   aa) + (b^* \cap bb)$.
\end{eg}


\subsection{Expression partial derivative}

In definition 2 of \cite{pdere}, the partial derivative of a regular expression
$\dda(r)$ is specified. We repeat the definition here, with one modification,
and with the addition of expression repetition.

\begin{defn}[Expression partial derivative]
   \label{defn-pd-eset}
   The partial derivative defined as sets of expression sets.

   \[\nrdef{1} \dda(\phi) = \dda(\varepsilon) = \dda(b) = \{\{ \phi \}\}\]

   \[\nrdef{2} \dda(a) = \{\{ \varepsilon \}\}\]

   \[\nrdef{3} \dda(r_1 + r_2) = \dda(r_1) \cup \dda(r_2)\]

   \[\nrdef{4} \dda(r^*) = \dda(r) \circledcdot r^*\]

   \[\nrdef{5} \dda((r_1, r_2)) =
       \begin{cases}
          \dda(r_1) \circledcdot r_2 & \text{if } \varepsilon \not\in L(r_1\esetsOfRe) \\
          \dda(r_1) \circledcdot r_2 \cup \dda(r_2) & \text{otherwise}
       \end{cases}
   \]

   \[\nrdef{6} \dda(r_1 \cap r_2) = \dda(r_1) \circledcap \dda(r_2)\]

   \[\nrdef{7} \dda(r^n) =
       \begin{cases}
          \dda(r) & \text{if } n = 1 \\
          \dda((r, r^{n-1})) & \text{otherwise}
       \end{cases}
   \]

   \[\nrdef{8} \dda(\neg r) = \circledneg (\dda(r))\]

\end{defn}

Caron et al. define \nr{1} as $\dda(b) = \emptyset$, but this yields incorrect
results when combined with negation. Consider for example the regular expression
$\neg b$, which matches anything that is not a `b', such as ``a''.  The partial
derivative over `a' according to \cite{pdere} would be constructed as follows:

\[\dda(\neg b)
   =_{\nr{8}} \circledneg \dda(b)
   =_{\nr{1}} \circledneg (\emptyset)
   = \emptyset
\]

The correct partial derivative is $\dda(\neg b) = \{\{\Sigma^*\}\}$. With our
definition, we construct the partial derivative as follows, considering that the
negation of the empty language is every possible word made from any number of
letters in the alphabet $\Sigma$:

\[\dda(\neg b)
   =_{\nr{8}} \circledneg \dda(b)
   =_{\nr{1}} \circledneg (\{\{\phi\}\})
   = \{\{\neg\phi\}\}
   = \{\{\Sigma^*\}\}
\]


% vim:tw=80
