\chapter{Introduction}
\pagenumbering{arabic}
\setcounter{page}{1}

Regular expressions, often abbreviated as ``regex'', are a key component of
modern software development. Many applications nowadays are inconceivable
without the efficient pattern matching provided by regular expression engines.

Regexes are used in NIDS\footnote{Network Intrusion Detection System} to match
network packets in real time, in compilers to tokenise source code, in online
search engines, and many other fields of everyday computing. It is often
required to not only decide whether a pattern matched an input, but also to
extract information from it. This is where pattern submatching comes into play.

Pattern submatching is the extraction of matched subexpressions from the input.
E.g. given the input \textit{``My name is Jack''} and pattern \textit{``My name
is (x:Jack$|$Jill)''}\footnote{The $|$ operator means ``or''.}, we want to know
which name, Jack or Jill, was provided, so we can retrieve the submatch
information stored in $x$ and do further processing without having to re-parse
the input.

It is apparent that regular expression matchers should be both efficient and
robust, so ever more clever optimisations and ever more features are implemented
in engines used by Perl, Java and many other programming languages. Most
programming languages have at least one engine written in it, or have bindings
to a popular C regular expression engine, such as PCRE\footnote{Perl Compatible
Regular Expressions}.

However, most of these engines share a common deficiency: they are slow.
Naturally, they are fast for most common expressions because that is what they
were optimised for, but for almost every regular expression engine, a class of
expressions exists for which the engine is slow. Slow, in this context, means
that it takes exponential execution time, and that means that even for
relatively small inputs, the time taken for a match can exceed a human lifetime.

C\# can compile regular expressions to CIL\footnote{Common Intermediate
Language, the byte code used by the Microsoft .NET Framework} byte code, which
is then compiled to native code by the .NET Just-In-Time compiler. PCRE contains
a JIT\footnote{Just-In-Time compilation: generating machine code just before
execution} mode, as well, but these optimisations, while very useful, only
reduce the runtime by a constant factor.

Why do these engines have such inefficient edge-cases? The reason is that all of
them implement a superset of regular languages. In order to implement these
non-regular expressions, backtracking algorithms are used.

The fact that these algorithms take exponential runtime make them unfeasible for
many applications, in particular those that allow untrusted users to enter
arbitrary expressions. Russ Cox developed RE2, an automata based regular
expression engine, for exactly this reason: Google code search allows web users
to enter regular expressions that run against a large amount of source code. An
engine with potentially exponential runtime would impose a security risk.

Applications requiring linear execution time for pattern matching generally use
actual regular expressions, i.e. descriptions of regular languages, that can be
compiled to efficient matching automata.

There are several ways to construct such automata. A well-known method is
Thompson's construction algorithm, described in \cite{thompson}. The paper
mentions backtracking algorithms and explains how finite automata can improve
execution time from exponential to linear. It builds on ideas from Janusz
Brzozowski's, who, in his article \cite{brzozowski}, developed a formalism to
construct a finite automaton for a regular language by taking the left
derivative of the language with respect to its input.


\section{Key ideas}

This master's thesis presents an increment to the theory of regular expression
matching. The ideas this work extends is mostly covered in the following works:

\begin{itemize}
      
   \item In \cite{pdpat}, Martin Sulzmann and Kenny Zhuo Ming Lu incorporate
      pattern submatching with partial derivatives.

      Partial derivatives are an elegant method of constructing an efficient
      nondeterministic finite automaton. Section \ref{pd} explains this concept
      in detail.

   \item Scott Owens, John Reppy and Aaron Turon introduce extended regular
      expressions to derivatives in \cite{re-deriv}.

      Extended regular expressions are regular expressions with additional
      boolean operators $\cap$ (intersection) and $\neg$ (negation). The theory
      of this paper was implemented in \mlulex, part of the
      SML/NJ\footnote{Standard ML of New Jersey} distribution. To our knowledge,
      it is the only widely available tool that supports this type of extended
      regular expressions.

   \item Pascal Caron, Champarnaud and Mignot describe a method of extended
      regular expression matching with partial derivatives that does not reduce
      to a derivative computation\cite{pdere}.

      They present a very elegant way to mathematically define the partial
      derivative of an extended regular expression.

\end{itemize}

In particular, our contributions are:

\begin{itemize}

   \item Pattern submatching based on partial derivatives with extended regular
      expressions. We use a variation of the method presented by Caron et al.
      combined with Sulzmann and Lu's pattern submatching.

   \item Clarification of some of the ideas presented in the papers this thesis
      is based on.

   \item A generic submatching engine based on partial derivatives and
      incorporating extended regular patterns.
      
      This engine is written in OCaml, so it may more easily be compared to the
      existing \mlulex{} tool. Sulzmann and Lu implemented their submatching
      engine in Haskell, a lazy purely functional language, whereas \mlulex{}
      was written in SML, an eager impure functional language. OCaml, also being
      an eager impure functional language, gives us the opportunity for an
      expressive performance evaluation.

   \item A lexer tool compatible with \ocamllex\footnote{The lexer generator
      shipped with OCaml.} built with the generic engine.

\end{itemize}


\section{Efficient submatch extraction}

The goal of this thesis is to create an efficient replacement for \ocamllex{}
based on the theory of partial derivatives. OCaml contains the \ocamllex{}
program, which is used in the implementation of the OCaml compiler itself, as
well as many other languages implemented in OCaml.  Unfortunately, this tool has
several serious drawbacks:

\begin{itemize}

   \item The automaton is restricted in size.

      It is not possible to list all C keywords in the lexer description, since
      the program will refuse to generate a large automaton.

   \item Expression syntax is weak.

      There is no support for regular expression negation or intersection.
      Counted repetition is not supported. The accepted expression syntax is
      very basic.

   \item No compositing.

      This is not a serious drawback, but parser generators such as Menhir have
      supported grammar compositing for a long time. Lexical grammar compositing
      is particularly useful when creating a combined C/C++ front-end or for
      supporting language extensions such as GNU C keywords while keeping the
      base lexer definition clean and standard.

   \item Weak support for line number counting.

      In order to correctly track the line number in the lexer, the user must
      create a special rule in every sub-automaton where a new-line character
      may occur, just to call \texttt{Lexing.new\_line lexbuf}. This requirement
      makes the resulting lexer slower and the grammar more error-prone.

\end{itemize}

In the course of this thesis, we developed a new tool named
\dreml\footnote{(Partial) \textbf{D}erivatives based \textbf{R}egular
\textbf{E}xpressions for \textbf{ML}}, which is fully backwards compatible with
\ocamllex, but solves all of the above drawbacks.

\begin{itemize}

   \item The automaton size is unbounded.

      The presented approach is based on non-deterministic finite automata,
      constructed from partial derivatives. The automaton size is therefore
      linear in the expression length. A lexer generated by \dreml{} from a
      grammar twice as long will be twice as large. The tool can easily support
      thousands of keywords without the automaton growing to impractical sizes.

   \item Pattern submatching is fully integrated.

      Tagged transitions do not considerably increase the automaton size. There
      is no limit to the number of submatches. In fact, rules in \dreml{} are
      implemented as submatches, internally.

   \item Richer expression syntax.

      The integration of extended regular expression syntax, as well as some
      other commonly found extensions, give \dreml{} users the power to describe
      their lexical grammar in a more expressive way.

   \item Grammar compositing.

      \dreml{} has direct support for importing definitions and even entire
      grammars from other files.

   \item Direct support for line numbers.

      Every transition that can occur on the new-line character automatically
      calls \texttt{Lexing.new\_line}, so the user can omit these special rules
      entirely. For backwards compatibility, this support can be disabled.

\end{itemize}


\section{Pattern submatching}

The C programming language specifies integer literals as a number followed by
zero or more type suffixes, \texttt{l} standing for \texttt{long}, \texttt{u}
for \texttt{unsigned}. In a C compiler, it is therefore useful to extract the
parts directly during the lexical analysis, in order to reduce duplication of
knowledge.

Perhaps an even better example for a case where pattern submatching is useful is
in C preprocessor directives, particularly the \verb!#include! directive. A
parser or preprocessor is only interested in the name of the included file,
which can be extracted using a pattern such as $\texttt{\#include } W^*
\texttt{"} (f:(\Sigma \setminus \{\texttt{"}\})^*) \texttt{"}$.

Matching a valid include-directive with this pattern will record the file name
in the pattern variable $f$. E.g. considering input \verb!#include "stdio.h"!,
the matching environment will consist of the set $\Gamma = \{
(f:\texttt{stdio.h}) \}$. The file name can then be extracted and used in a
semantic action or post-processing step.

\subsection{Declarative lexer specifications}

Lexical analysers for programming languages are often described using domain
specific lexer specification languages. These consist of a list of regular
expressions associated with semantic actions, which are most often written in
the target language. Tools such as \texttt{lex} and derivatives for other
languages read these specifications and produce efficient matching automata from
them.

After a lexeme has been recognised, the lexer needs to call the appropriate user
code that will produce a token or send the information to an asynchronous
communication channel. In general, a lexer function receives an input stream
that reads characters until a single token has been successfully read and
returns. It must handle multiple regular expressions in parallel, and the
semantic action associated with the longest left-most match is executed.

Lexer tools often produce a large regular expression with all sub-expressions in
disjunction. In other words, all expressions $r_1 \dots r_n$ are put into a
unifying choice $(r_1 + \dots + r_n)$. The last final state and the stream
position are saved and restored when the automaton goes into the error state.
When the error state is reached, either by reading the end of file symbol or
because none of the regular expressions can continue with the symbol read, the
code associated with the last final state is executed and the stream position is
restored.

\cite{re-deriv} uses Brzozowski's regular vectors to handle multiple regular
expressions in the context of scanners when using regular expression
derivatives. This method does not create a disjunction over all expressions, but
rather preserves each of them in a vector, so that the expressions $r_1 \dots
r_n$ form the regular vector $(r_1, \dots, r_n)$. The derivative operation is
performed component-wise, so that $\dda(r_1, \dots, r_n) = (\dda r_1, \dots \dda
r_n)$. DFA\footnote{Deterministic Finite Automaton} states are labelled with
regular vectors instead of regular expressions. A final state is defined as a
state labelled with a vector containing at least one regular expression $r$ with
$\varepsilon \in L(r)$, that is, it accepts the empty word, and the error state
is the one in which every vector member $r$ represents the empty language, that
is $L(r) = \emptyset$.

We can recognise that the goal achieved by using regular vectors is very similar
to submatching. In fact, the problem can easily be described in terms of
submatching. Using a generic submatching engine, one can elegantly declare a
scanner for lexeme patterns $(x_1:r_1) \dots (x_n:r_n)$ as a disjunction
$((x_1:r_1) + \dots + (x_n:r_n))$. The greedy left-most match then yields the
same result as the regular vector or the ad-hoc state switch approach.

We implemented a prototype of this method in the tool \texttt{drelex}, part of
\dreml.


\section{Extended regular expressions}

This thesis combines pattern submatching with extended regular expressions,
i.e. including the additional boolean operators $\neg$ for negation and $\cap$
for intersection. Expression negation is useful for C comments of the form
\verb!/* ... */!. A pattern matching C comments and extracting the comment text
may be written as
\[
  \texttt{/*}(x:\neg(\Sigma^*\texttt{*/}\Sigma^*))\texttt{*/}
\]
Describing the same language without negation would require a longer
expression\footnote{Taken from \cite{re-deriv}.}:
\[
  \texttt{/*}
  (
    (\Sigma \setminus \{*\})^*
    (\varepsilon + \texttt{*}^* (\Sigma \setminus \{\texttt{/}, \texttt{*}\}))
  )^*
  \texttt{*/}
\]
Patterns involving more complex or more lengthly opening and closing expressions
would become much more cumbersome to express without negation.

Intersections are useful in subtracting regular expressions from patterns. One
could specify a base pattern for C integer literals, not including hexadecimal
literals, as $p_{\mathit{int}} = (\mathit{num}:(0 \dots 9)^n)
(\mathit{suf}:(l+L+u+U)^*)$, where $n \in \mathbb{N} \setminus \{0\}$, after
which \textit{num} contains the number and \textit{suf} the type suffix. Next,
we can restrict the pattern for octal integer literals by requiring it to begin
with a zero followed by anything not containing digits 8 or 9: $p_{\mathit{oct}}
= p_{\mathit{int}} \cap (0, \Sigma \setminus \{8, 9\})$.

This is particularly useful in the presence of composed regular grammars. A
library of standard regular expressions may define a set of valid C identifiers,
which may then be restricted in specialised lexers used to verify a coding style
or perform syntax highlighting based on coding conventions.


\section{Outline}

\begin{description}

   \item[Chapter \ref{background}] will explain the background for the ideas on
      which the developed tool is based. A basic regular expression syntax is
      defined and later extended with additional boolean operators. Brzozowski's
      derivatives will be outlined in the first section, which is followed by a
      short introduction to Antimirov's partial derivatives. In the next
      section, pattern submatching as done by Sulzmann and Lu in \cite{pdpat} is
      introduced in some detail. Finally, a representation of partial
      derivatives as disjunctive normal form of terms as described in
      \cite{pdere} is explained and shown by example.

   \item[Chapter \ref{submatching}] aims to combine the two ideas and introduces
      extended pattern submatching. A pattern matching relation for extended
      regular expression patterns is specified, and in compliance with that, the
      partial derivative operation is defined. Section \ref{nfa} uses this
      definition to provide a pattern partial derivative function compatible
      with the one in \cite{pdpat}, and extended by the syntax of \cite{pdere}.
      It also explains how this function can be used to construct a finite state
      machine from a pattern. At the end of this chapter, a language predicate
      optimisation is defined in order to reduce the number of unreachable
      states produced by the automata construction.

   \item[Chapter \ref{implementation}] describes the implementation details of
      the generic submatching engine, which does not support all of \dreml's
      input syntax, but implements all the required algorithms. The prototype
      first implemented Brzozowski's derivatives, then Antimirov's partial
      derivatives, which were enhanced by Sulzmann and Lu's pattern submatching
      and finally extended with intersection and negation. This development
      process will be followed and peculiarities pointed out. After the basic
      implementation has been described in detail, some of the more important
      optimisations will be described.

   \needspace{2cm}
   \item[Chapter \ref{results}] follows the implementation, presenting empirical
      results in the form of timing and comparison with existing tools and
      algorithms. We compare our implementation with the \mlulex{} tool and
      evaluate both compile time of the expression and runtime of the matching
      automaton.

   \item[Chapter \ref{discussion}] discusses the results of this thesis, as well
      as some results of related work. It draws the distinction between the
      presented work and previous results. In this chapter, we also clarify the
      difference between what POSIX calls ``Extended Regular Expressions'' (ERE)
      and what we understand them to be. This is to avoid confusion to a reader
      more accustomed to the terminology commonly found in manuals of tools
      based on POSIX regular expressions.

   \item[Chapter \ref{conclusion},] the conclusion, reiterates the presented
      ideas, briefly summarises the results, and suggests some directions in
      which future development could go. A link to the prototype implementation
      on which our ongoing development will take place is also provided at the
      end of this chapter.

\end{description}


% vim:tw=80
