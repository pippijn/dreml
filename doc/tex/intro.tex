\chapter{Introduction}

Regular expressions are a key component of modern software development. Many
applications are nowadays inconceivable without the efficient pattern matching
provided by regular expression engines. Ever more clever optimisations and ever
more features are implemented in engines used by Perl, Java and many other
programming languages. Most programming languages have at least one engine
written in it, or have bindings to a popular C regex engine such as PCRE.

Many of these matching engines support a superset of regular languages. In order
to implement these non-regular expressions, engines use backtracking algorithms.
The fact that these algorithms take exponential runtime make them unfeasible for
many applications, such as compilers and public search engines. Applications
requiring linear execution time for pattern matching generally use real regular
expressions, i.e. descriptions of regular languages, that can be compiled to
efficient matching automata.

There are several ways to construct such automata. A well-known method is
Thompson's construction algorithm, described in \cite{thompson}. The paper
mentions backtracking algorithms and explains how finite automata can improve
execution time from exponential to linear. It builds on ideas from Janusz
Brzozowski's, who, in his article \cite{brzozowski}, developed a formalism to
construct a finite automaton for a regular language by taking the left
derivative of the language with respect to its input.  Brzozowski in .

\begin{itemize}
   \item What is the thesis about?
   \item What problems will be solved?
   \item Outline.
   \item Key ideas.
\end{itemize}

When constructing the partial derivative of a regular expression, the choice
operator results in a set of possible derivatives, rather than a single
derivated term.


% vim:tw=80
