\section{Pattern matching relation}
\label{patmatchrel}

We write the union of pattern match environments as $\uplus$, which is, since
we use ordered multi-sets to represent an environment, equivalent to list
concatenation. We formalise the pattern matching relation $w \vdash p \leadsto
\Gamma$ based on the definition in \cite{pd-pat}, extending it with
intersection. We do not define a relation on pattern negation, since at this
point, these must have been put in the negation normal form using $\negnorm$.

\begin{minipage}[t]{0.4\textwidth}
   \[\text{(VarBase)} \frac
      {w \in L(r)}
      {w \vdash x : r \leadsto \{x : w\}}
   \]
\end{minipage}
\begin{minipage}[t]{0.4\textwidth}
   \[\text{(VarGroup)} \frac
      {w \vdash p \leadsto \Gamma}
      {w \vdash x : p \leadsto \{x : w\} \uplus \Gamma}
   \]
\end{minipage}

\begin{minipage}[t]{0.4\textwidth}
\[\text{(Concat)} \frac
   {\begin{array}{c}
      w = w_1 w_2 \\
      w_1 \vdash p_1 \leadsto \Gamma_1 \\
      w_2 \vdash p_2 \leadsto \Gamma_2
   \end{array}}     
   {w \vdash (p_1, p_2) \leadsto \Gamma_1 \uplus \Gamma_2}
\]
\end{minipage}
\begin{minipage}[t]{0.4\textwidth}
\[\text{(Star)} \frac
   {\begin{array}{c}
      w = w_1 \dots w_n \\
      w_i \vdash p \leadsto \Gamma_i \text{ for } i = 1 \dots n
   \end{array}}     
   {w \vdash p^* \leadsto \Gamma_1 \uplus \dots \uplus \Gamma_n}
\]
\end{minipage}

\begin{minipage}[t]{0.4\textwidth}
\[\text{(ChoiceL)} \frac
   {w \vdash p_1 \leadsto \Gamma_1}
   {w \vdash p_1 + p_2 \leadsto \Gamma_1}
\]
\end{minipage}
\begin{minipage}[t]{0.4\textwidth}
\[\text{(ChoiceR)} \frac
   {w \vdash p_2 \leadsto \Gamma_2}
   {w \vdash p_1 + p_2 \leadsto \Gamma_2}
\]
\end{minipage}

\[\text{(Intersect)} \frac
   {\begin{array}{c}
      w \vdash p_1 \leadsto \Gamma_1 \\
      w \vdash p_2 \leadsto \Gamma_2
   \end{array}}     
   {w \vdash p_1 \cap p_2 \leadsto \Gamma_1 \uplus \Gamma_2}
\]

As described in section \ref{ere-submatch}, the matching function yielded by
the pattern partial derivative can be customised with an $\iterate$. If this
function is defined as one that keeps all matches, the number of possible
matches can become exponential, if the pattern matching relation is defined as
above.

We consider the pattern $(x:A^*)$ and the input $A^n$. This pattern has one
possible match, which is the complete input $A^n$. If we now add an iterating
operator $p^*$ to the pattern, we match with the pattern $(x:A^*)^*$. The
number of possible matches is now $2^{n-1}$, namely all combinations of the
set $\{ A^1, \dots, A^n \}$ that yield the original input $A^n$.

If the Kleene star is applied again, the number of matches does not change,
but the number of states the automaton ends with will be $m^{n-1}$ where $m$
is the number of kleene stars in tetration: $S((x:A^*)^{^m*}, A^n) = m^{n-1}$.

This exponential number of matches can be reduced if instead of an ordered
multi-set of matches, we use an unordered multi-set for the matching
environment. E.g. the match $\{ x:A; x:AA \}$ and $\{ x:AA; x:A \}$ will be
equivalent and counted only once. The resulting match-sets will be the set of
parts of all partitions of $n$ represented as $A^n$. Consequently, the number
of match-sets will be equal to the cardinality of the set of sets of parts of
all partitions of $n$. % This number is polynomial. TODO: really? and how many?

We can reduce the number of matches to one by defining a deterministic pattern
matching relation. The most commonly used disambiguation strategy used is the
greedy left-most matching strategy as used by Perl style regular expression
engines. A different strategy is used by POSIX, which favours the longest word
match. We employ the Perl strategy, which is natural for partial derivatives,
as we can simply take the first matched variable from the ordered multi-set
that is the matching environment.



% vim:tw=78
