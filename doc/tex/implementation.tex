\chapter{Implementation}
\label{implementation}

A prototype implementation was done in OCaml, a functional programming language
with imperative and object-oriented elements. This chapter describes how the
prototype was initially built for derivatives, how it was changed to partial
derivatives, how support for extended regular expressions was implemented, and
finally how the prototype was optimised.

In addition to the prototype, the algorithms implemented were integrated into a
tool called \dreml, which accepts the full syntax supported by \ocamllex,
including submatching.

\img{modules}{10cm}{Module dependency graph}


\section{Grammar}
%%%%%%%%%%%%%%%%%

The lexical grammar in the prototype supports only Latin letters, $a$ through
$z$ in upper and lower case, for matches. This limitation is rather arbitrary
but removes the need for escape sequences. The \dreml{} tool supports any
character, as well as Unicode escape sequences, which are translated to a UTF-8
encoded sequence.

\begin{defn}[Grammar of regular expression syntax]
   \label{defn-gram-reg}
   Regular grammar for tokens used in the context-free grammar below.

   \begin{tabular}{lrl}
      Digit	& $::=$	& '0' \dots '9'				\\
      Number	& $::=$	& Digit Number				\\
      Character	& $::=$	& 'a' \dots 'z' $|$ 'A' \dots 'Z'	\\
      Namechar	& $::=$	& 'a' \dots 'z'				\\
      Namestr	& $::=$	& Namestr Name				\\
      Name	& $::=$	& Namestr ':'				\\
   \end{tabular}
\end{defn}

The \textit{name} terminal includes the trailing ':' so that it can be
distinguished from a single character. Another way would be to allow
\textit{namestr} both as name and as expression, and transforming the string
into a character sequence at a later point.

\begin{defn}[Context-free grammar of regular expression syntax]
   \label{defn-gram-cfg}

   In this concrete syntax definition, identifiers starting with a
   majuscule\footnote{Uppercase letter} name tokens defined in the lexical
   grammar, and identifiers starting with a minuscule\footnote{Lowercase letter}
   name other context-free grammar rules defined here.

   \needspace{5cm}

   Patterns:

   \begin{tabular}{lrl}
      pattern		& $::=$	& pat\_and			\\
      pat\_and		& $::=$	& pat\_branch			\\
      			& $|$	& pat\_and '$|$' pat\_branch	\\
      pat\_branch	& $::=$	& pat\_concat			\\
      			& $|$	& pat\_and '\&' pat\_concat	\\
      pat\_concat	& $::=$	& pat\_expr			\\
			& $|$	& '$\sim$' pat\_expr		\\
			& $|$	& pat\_concat pat\_expr		\\
      pat\_expr		& $::=$	& '(' Name regex ')'		\\
			& $|$	& '(' Name pattern ')'		\\
			& $|$	& '(' pattern ')'		\\
			& $|$	& pat\_expr '*'			\\
                        & $|$	& pat\_expr '\verb!^!' Number	\\
                        & $|$	& pat\_expr '\{' Number '\}'	\\
   \end{tabular}

   \needspace{5cm}
   Expressions:

   \begin{tabular}{lrl}
      regex		& $::=$	& re\_and			\\
      re\_and		& $::=$	& re\_branch			\\
      			& $|$	& re\_and '$|$' re\_branch	\\
      re\_branch	& $::=$	& re\_concat			\\
      			& $|$	& re\_and '\&' re\_concat	\\
      re\_concat	& $::=$	& re\_expr			\\
			& $|$	& '$\sim$' re\_expr		\\
			& $|$	& re\_concat re\_expr		\\
      re\_expr		& $::=$	& '(' regex ')'			\\
			& $|$	& re\_expr '*'			\\
			& $|$	& re\_expr '?'			\\
                        & $|$	& re\_expr '\verb!^!' Number	\\
                        & $|$	& re\_expr '\{' Number '\}'	\\
                        & $|$	& Character			\\
   \end{tabular}
\end{defn}

The parser also allows for '$\cap$' (U+2229) to be used as intersection operator
'\&', as well as '+' for the choice operator '$|$' and '$\neg$' (U+00AC) for
negation '$\sim$'. The two syntaxes for counted repetition are equivalent.  The
repetition operator is not supported in \ocamllex, but \dreml{} allows
it. It should be noted that since a finite state machine does not have any
storage besides the states it consists of, the repetition operator yields the
same automaton as an expanded expression with the expression under repetition
actually repeated $n$ times.  Thus, the number of states required by the
automaton grows linearly with $n$.


\section{Derivatives}
%%%%%%%%%%%%%%%%%%%%%

In a first step after defining the abstract and concrete syntax for regular
expressions, a derivative operation was defined that computes the derivative of
a pattern over one input symbol.

The derivative function operates directly on the syntax tree structure of the
parsed expression and yields a value of the same type. The $\nullable$ predicate
is used in the implementation of the derivative of regular expression
concatenation.

In the initial implementation, the syntax tree contained not only the pattern
variable name, but also its value. Thus, the derivative operation directly
records the submatch in the resulting pattern derivative instead of yielding a
pattern derivative along with a transition function as discussed in section
\ref{exprsets}.

A consequence of this interpretative method is that an automaton cannot be
pre-computed or even cached. Every derivative is always computed from scratch.
The amount of memory needed by this method is the size of the largest expression
derivative in the automaton.

The pattern derivative over a word is then defined in terms of the pattern
derivative over a letter:

\begin{tabular}{lrl}
   $p \setminus \varepsilon$	& $=$ & $p$				\\
   $p \setminus lw$		& $=$ & $(p \setminus l) \setminus w$	\\
\end{tabular}

This definition is from \cite{pdpat}, figure 7.

\begin{eg}
   Computation of pattern derivative of a word.

   Given the pattern $(x|[\:]:AB)+(y|[\:]:AC)+(z|[\:]:ABC)$ and the input string $AB$,
   the derivative of the pattern over the word is recursively constructed.
   \begin{align*}
        &\quad (x|[\:]:AB)+(y|[\:]:AC)+(z|[\:]:ABC) \setminus A \setminus B \\
      = &\quad (x|[A]:B)+(y|[A]:C)+(z|[A]:BC) \setminus B \\
      = &\quad (x|[AB]:\varepsilon)+(y|[AB]:\phi)+(y|[AB]:C)
   \end{align*}
\end{eg}

In order to obtain the greedy left-most match, we need to do a depth-first
left-to-right search on the resulting pattern structure, looking for
sub-patterns consisting of $\varepsilon$, extracting at most one of each pattern
variable. This process will, in the above example, ignore the failed match
$(y:AC)$, as well as the incomplete match $(z:ABC)$. Sub-patterns of failed
matches are ignored, which means the patterns at the highest level of the syntax
tree must succeed for all others to yield a value.


\section{Partial derivatives}
%%%%%%%%%%%%%%%%%%%%%%%%%%%%%

The partial derivative operation producing a list of derivated terms instead of
the complete derivative. This is done by returning a list of terms where the
derivative operation returns a choice. The derivative operation for choice is
defined as $(p_1 + p_2) \setminus l = p_1 \setminus l + p_2 \setminus l$. For
the partial derivative, we return the choices separately as a list, and since
each partial derivative is itself a list, the definition is changed to $(p_1 +
p_2) \setminus_p l = p_1 \setminus_p l ++ p_2 \setminus_p l$, where $++$ is the
list concatenation or in other words, ordered set union, operator.

At this point, we still use the expression structure to record matches, so the
partial derivative of the above example would yield:

\begin{eg}
   Computation of pattern partial derivative of a word.
   \begin{align*}
        &\quad (x|[\:]:AB)+(y|[\:]:AC)+(z|[\:]:ABC) \setminus_p A \setminus_p B \\
      = &\quad \{ p' \setminus_p B | p' \in \{(x|[A]:B), (y|[A]:C), (z|[A]:BC)\} \} \\
      = &\quad \{ (x|[AB]:\varepsilon), (y|[AB]:\phi), (z|[AB]:C) \}
   \end{align*}
\end{eg}

Here, we can obtain the greedy left-most match by taking the first succeeding
pattern from the final list and extracting all matches from that. Since the
partial derivative contains no choice operators, there can be no clashes in
variable names due to several paths being taken.

A Brzozowski derivative from the Antimirov partial derivative constructed here
can easily be obtained by folding the set of derivated terms into choices:

\[p \setminus l = (p_1 + \dots + p_n) \where p_i \in p \setminus_p l \]

This function is used to implement negation and intersection as a natural
extension to partial derivatives. It is not used in the sets of sets extension.


\subsection{Transition function}

The problem with recording the match in the pattern structure is still that we
cannot pre-compute a matching automaton. For this reason, the derivative
operation must yield a derivative along with a transition function.

In our prototype implementation, the transition function is described by the
abstract type \texttt{TransitionType.t} in module \texttt{Types}.

\begin{lstlisting}
   (* Transition function interface. *)
   module type TransitionType = sig
     type t

     val update : (string * letter) -> t
     val iterate : pattern -> t
     val compose : t -> t -> t

     val execute : t -> env -> env

     val show : t -> string
   end
\end{lstlisting}

Abstract type \texttt{t} represents the transition function. Semantically, it
represents a function of type $\mathit{env} \to \mathit{env}$. In the concrete
implementation \texttt{Transition}, that is exactly its type, and function
composition is done with function closures. Function \texttt{Transition.execute}
is then simply a function application operator. The implementation in module
\texttt{Instruction} defines it as a data structure that is interpreted by
\texttt{execute}.

Ultimately, the data structure based execution calls the functions from the
\texttt{Transition} module, and is therefore slightly less efficient than a
direct closure, but has the advantage that it can be inspected, whereas function
closures are effectively opaque. Furthermore, a data structure representing the
transition function can be used for code generation, so that our tagged
automaton can be translated to OCaml functions.

In our prototype, type \texttt{letter} is an alias for built-in OCaml type
\texttt{char}, but can naturally be extended to any input symbol type
representing the alphabet $\Sigma$.

In the first prototype, the built-in \texttt{string} type or the standard
library module \texttt{Buffer} were not used, in order to keep the
implementation close to the mathematical definition, which is convenient when
using proof assistants to investigate the program. Section \ref{opts} describes
how different data structure choices affect various performance characteristics
of automaton simulation.


\subsubsection{compose}

The first three functions are constructors for the transition type, two of which
correspond to operations required in definition \ref{defn-pd-pset}. A third
function, \texttt{compose}, resembles the mathematical function composition
operator $\circ$, so that $(f \circ g)(\mathit{env})$ can be directly mapped to
the OCaml code \texttt{execute (compose f g) env}.


\subsubsection{instruction type}

Type \texttt{instruction} is used in module \texttt{Instruction} to represent
the transition function as data structure.

\begin{lstlisting}
   (* Transition function as data structure. *)
   type instruction =
     | Update of string * letter
     | Iterate of string list
     | Compose of instruction * instruction
\end{lstlisting}


\subsubsection{iterate}

\texttt{iterate} is a two-step function. In the definition of pattern partial
derivatives, $\iteratefvp$ is indexed by $\fv(p)$, which is the set of pattern
variables in $p$. When calling \texttt{iterate}, these variables are renamed, so
that the next iteration of $p^*$ or $p^n$ receives a pristine matching
environment.

Since the value of $\fv(p)$ does not depend on the input, it can be computed
once and stored in the transition function. \texttt{Transition} stores the set
of variables in the closure, and \texttt{Instruction} explicitly stores it in
the string list of variant \texttt{Iterate}.


\subsubsection{update}

Finally, the most important function, \texttt{update (x, l)} records the fact
that a letter $l$ has been matched by pattern variable $x$. The matching
environment is a list of pairs containing name and value of matches. Type
\texttt{env} is \texttt{(string * word) list}, where type \texttt{word = char
list}.


\subsubsection{show}

Every data structure used in the program can be converted to a string
representation. We use the OCaml library \texttt{Deriving\_Show} to generate
functions for every type, and for some types, we provide specialised functions
that visualise data on a higher semantic level. Since the captured environment
of function closures cannot normally\footnote{Without using the \texttt{Obj}
interface} be inspected in OCaml, \texttt{Transition.show} does nothing, so
\texttt{Instruction.show} can be used to visualise the transition function's
effects on a passed matching environment.


\subsection{Parametrised partial derivative function}

In order to support both concrete implementations of the transition function
interface, module \texttt{Partial}, which exports the entry point of the partial
derivative operation, is parametrised by \texttt{TransitionType}. Only the
pattern partial derivative function requires this parametrisation; the
expression partial derivative is outside the functor definition.

\needspace{3cm}
\begin{lstlisting}
   (* Partial derivative interface. *)
   module type PartialDerivative
     : functor(Tag : TransitionType) ->
   sig
     val derive_pat
       : letter
       -> pattern
       -> (pattern * Tag.t) list
   end
\end{lstlisting}


\subsection{Automata construction}

When constructing the automaton, a state is represented by the structure of its
pattern. Each state is associated with a dense map of input symbol to a set of
states it can transition to on that input symbol. We thus represent the
non-deterministic finite automaton $(Q, \Sigma, \Delta, q_0, F)$ as

\begin{lstlisting}
   type state = pattern
   type transition = (state * instruction) list
   type nfa = (state, transition array) map * state
\end{lstlisting}

In this definition,

\begin{itemize}
   \item $Q$ is the set of states represented by map's keys.
   \item $\Sigma$ is the set of bytes between 0 and 255.
   \item $\Delta$ is the matching relation $Q \times \Sigma \to \mathcal{P}(Q)$,
      in which $Q$ is the key \texttt{state}, $\Sigma$ is used to index in the
      mapped \texttt{transition array}, and $\mathcal{P}(Q)$ is the
      \texttt{transition}, which also contains a matching function for updating
      the matching environments.
   \item $q_0$ is the second component of the \texttt{nfa} pair.
   \item $F$ is implicitly given by $F = \{ q \in Q \where \varepsilon \in L(q) \}$.
\end{itemize}

A state lookup takes $O(n)$ time over the term length and amortised $O(1)$ over
the number of states ($O(n)$ worst case) with a hash table, or $O(\log n)$ when
using a tree based map structure.


\subsection{Matching strategies}

Without a matching strategy, the engine will collect every possible match for
every variable. E.g. $(x:a*)(y:a*)$ for $aa$ will yield $\{[(x:aa)], [(x:a),
(y:a)], [(y:aa)]\}$. For a longer sequence of $a$, the set of results will grow
exponentially.

Each unique NFA state can contain a different environment, which is carried
along the path the thread takes. When multiple paths come together, it is the
matching strategy that decides which of the environments is retained. For
example, considering the pattern $p = (x:a^*) + (y:aa)$ and input string $aa$,
we would first transition over $a$, which yields the partial derivative $p
\setminus_p a = \{ ((x:\varepsilon), (x \mapsto a)), ((x:a^*), (x \mapsto a)),
((y:\varepsilon a), (y \mapsto a)) \}$. Here, the first component of each pair
is the expression matching at the next state, and the second component is the
associated matching environment $\Gamma$. In the next step, we derive each of
these terms over the next $a$, amending their respective environments, yielding
a set of new states for each of the three states in $p'$.

\begin{enumerate}
   \item
      $(x:\varepsilon) \setminus_p a = \{ ((x:\phi), (x \mapsto a)) \}$
   \item
      $(x:a^*) \setminus_p a = \{
         ((x:\varepsilon), (x \mapsto a)),
         ((x:a^*), (x \mapsto a))
      \}$
   \item
      $(y:\varepsilon a) \setminus_p a = \{ ((y:\varepsilon), (y \mapsto a)) \}$
\end{enumerate}

These sets are now merged into a single ordered set, while updating the input
environments, yielding $p'' = \{ ((x:\varepsilon), (x \mapsto aa)), ((x:a^*), (x
\mapsto aa)), ((y:\varepsilon), (y \mapsto aa)) \}$. In the process of merging
the resulting sets, we ignore all states that have reached $\phi$, and are
therefore in error.

Now, two threads arrived in the same state: $(x:\varepsilon)$ and
$(y:\varepsilon)$. A matching strategy decides which one is kept. According to
the greedy left-most matching strategy, that is the first, so that we end up
with the state-set $\{ ((x:\varepsilon), (x \mapsto aa)), ((x:a^*), (x \mapsto
aa)) \}$. In order to obtain the final environment, we select the left-most
final term, where a final term is one where $\varepsilon \in L(p)$, meaning it
accepts the empty word.


\section{Extended regular expressions}
%%%%%%%%%%%%%%%%%%%%%%%%%%%%%%%%%%%%%%

Support for additional boolean operators was implemented using the DNF extension
described in section \ref{exprsets}.  The \texttt{ExprSet} module provides the
improved submatching algorithm based on expression sets.  The relevant types
for the sets of expression sets and pattern sets are defined in the
\texttt{Types} module:

\begin{lstlisting}
   (* Types for sets of expression sets. *)
   type exprset = regex list
   type exprsets = exprset list

   type exprset_pat = pattern list
\end{lstlisting}

Sets of pattern sets are parametrised by their transition function type, and
must therefore be defined in the functor that defines the derivative operation.
Types \texttt{regex} and \texttt{pattern} are the abstract syntax tree types for
regular expressions and patterns. The internal \texttt{derive\_pat} function
returns a \texttt{(exprset\_pat * Tag.t) list}, which is converted by operator
$\reOfPset$ to match the interface type \texttt{(pattern * Tag.t) list}.




\section{Expression simplification}
%%%%%%%%%%%%%%%%%%%%%%%%%%%%%%%%%%%

In order to reduce the number of states, we can apply several simplifications to
the expression structure before and during the automata construction. Since
minimising an NFA is a very expensive operation, we prefer to reduce the
expression complexity directly.

Before starting the partial derivative automata construction and after each step
in the construction, the input terms are recursively simplified by the following
rules:

\begin{defn}[Simplification]
   \label{defn-simplify}
   Simplification rules for expressions and patterns.

   Patterns:

   \begin{tabular}{lrl}
      $\varepsilon p = p \varepsilon$	& $=$	& $p$				\\
      $p^{*^*}$				& $=$	& $p^*$				\\
   \end{tabular}

   \needspace{3cm}
   Expressions:

   \begin{tabular}{lrl}
      Intersection:								\\
      $r \cap r$			& $=$	& $r$				\\
      $\phi \cap r = r \cap \phi$	& $=$	& $\phi$			\\
   \end{tabular}

   \begin{tabular}{lrl}
      Concatenation:								\\
      $\phi r = r \phi$			& $=$	& $\phi$			\\
      $\varepsilon r = r \varepsilon$	& $=$	& $r$				\\
   \end{tabular}

   \begin{tabular}{lrl}
      Choice:									\\
      $r + r$				& $=$	& $r$				\\
      $\neg\phi + r = r + \neg\phi$	& $=$	& $\neg\phi$			\\
   \end{tabular}

   \begin{tabular}{lrl}
      Repetition:								\\
      $r^{*^*}$				& $=$	& $r^*$				\\
      $\varepsilon^*$			& $=$	& $\varepsilon$			\\
      $\phi^*$				& $=$	& $\varepsilon$			\\
      $r^0$				& $=$	& $\varepsilon$			\\
      $r^1$				& $=$	& $r$				\\
   \end{tabular}

   \needspace{2cm}
   \begin{tabular}{lrl}
      Negation:									\\
      $\neg(\neg r)$			& $=$	& $r$				\\
      $\neg(\neg r_1 \cap \neg r_2)$	& $=$	& $r_1 + r_2$			\\
      $\neg(\neg r_1 + \neg r_2)$	& $=$	& $r_1 \cap r_2$		\\
   \end{tabular}
\end{defn}

The last two negation rules are De Morgan's rules. All rules are implemented as
recursive rewrite rules, which means they must produce a strictly smaller
expression than they receive as input. A \texttt{rewrite} function recursively
reduces the expression using these rules, until the result converges. If a rule
produces a larger expression that may later be reduced by another rule, this
rewrite system may go into an endless loop. Therefore, all rules specified here
straightforwardly decrease term size.

\begin{rem}
   Simplification rules are not strictly required, and the automata construction
   can work correctly without them, if it uses ordered sets as representation of
   sets of pattern sets. However, since insertion into such sets is inefficient,
   we use simple lists (ordered multi-sets), and require simplification rules to
   ensure termination of the algorithm.
\end{rem}


\section{Optimisations}
%%%%%%%%%%%%%%%%%%%%%%%
\label{opts}

After the initial implementation was found to be very inefficient for various
obvious and non-obvious reasons, an optimised implementation was created, which
exhibits roughly five times better performance. The single most important
optimisation was the elision of all closures in the execution code. This
optimisation alone gained over 300\% performance. Instead of allocating a
closure and using it with general purpose higher-order functions such as
\texttt{map} and \texttt{fold}, all loops, the innermost of which is three
levels deep, were converted to simple recursive function calls. The cost of
allocating and calling a closure by far outweighs the cost of passing around all
required arguments, such as the current input position, states and
environments, to recursive calls.

Disabling OCaml runtime bounds checking in the arrays gained just about 3\%.
Approximately 8\% can be gained from using the inlining option in the OCaml
native compiler. A larger difference can be achieved by tuning the OCaml
generational garbage collector. The OCaml GC has a minor heap, in which recently
allocated objects reside, and a major heap for objects that survived a minor
garbage collection. Choosing an appropriate size for the minor heap so that
temporary lists created in several iterations can fit in it can drastically
improve performance. Experimental results show up to 34\% improvement for small
expressions with a minor heap size of between 8 and 32 kilobytes. Performing a
full collection of the minor and major heap, in which old objects reside, before
starting execution might help for very complex patterns and large inputs, but
its cost would outweigh any benefits for small inputs. Finally, structuring code
in inner modules costs 14\% performance on the OCaml implementation
used\footnote{OCaml version 4.00.1}.

Higher level optimisations are described in the following sub-sections.


\subsection{Automata construction}

Representing states as patterns is useful while computing the automaton, but is
inefficient during execution, since the map lookup is done symbolically over the
pattern structure.

Before starting NFA simulation, we first compute a more efficient representation
by numbering each pattern and replacing the symbolic lookup map with an array.
In this step, an inverse operation is also produced so that the simulation
result can be resolved back to a pattern. This numbering and mapping all
patterns in the original NFA data structure to their associated numbers yields
the following type of the NFA data structure:

\begin{lstlisting}
   type state = int
   type transition = (state * instruction) list
   type nfa = transition array * state
\end{lstlisting}

Every state lookup is now done through direct indexing into a single array,
making it $O(1)$ in both the number of states and the derived expression length.
A single\footnote{However, this method did not noticeably improve performance
against indexing into an array of arrays.} dense array is used for all states
for all possible input symbols. This means that indexing is done by $i = q \cdot
|\Sigma| + c$, where $q$ is the state number and $c$ is the input symbol number
in the ordered set of input symbols $\Sigma$.


\subsection{Matching strategies}

In our initial implementation, the set of next states and their associated
matching environments was first built and subsequently pruned in every state
transition. This allows for flexibility in the matching strategy, with which the
automata simulator could be parametrised. However, in subsequent optimising
iterations, the greedy left-most matching strategy was directly implemented in
the simulator loop. Instead of adding a duplicate state with its updated
environment and later removing it again, we keep track of the states already in
the transition list by adding its state number to a bit-map, and do not perform
a useless update and addition of a state already seen. This method clearly also
obtains the greedy left-most match. Other matching strategies could similarly be
implemented directly in the loop for improved performance.

For the bit-map, we use a byte array in the form of an OCaml \texttt{string}.
The first attempt was using a \texttt{bool array}, which is about twice as fast
for small automata, but suffers from cache misses when the automaton is larger,
at which point the byte string, despite slightly more complex access operations,
is more efficient. A second, non-negligible effect of using a string rather than
a boolean array is that the OCaml garbage collector knows not to scan it. This
effect is particularly noticeable for very large automata.


% vim:tw=80
