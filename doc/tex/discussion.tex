\chapter{Discussion and related work}
\label{discussion}

\section{Real world regular expressions}


\subsection{Extra syntax}

In this paper, we have discussed a minimal regular expression syntax, for the
sake of simplification. However, most applications using regular expressions
assume a superset of this syntax. In this section, we will discuss how popular
extensions to the regular expression syntax presented in this paper can be
transformed to that syntax.

\begin{itemize}

   \item We have discussed the Kleene star $r^*$, which allows any sequence of
      $r$ including the empty sequence. Often, it is necessary to have at least
      one $r$ in the sequence, which brings us to the Kleene plus $r^+$. This
      operator is implemented simply with the rewrite rule $r^+ \mapsto (r,
      r^*)$, i.e. an $r$ followed by any number of $r$ or nothing.

   \item Character classes, such as \verb'[a-z]' are often used to specify
      ranges of accepted symbols. For instance, an identifier or type name in
      OCaml might be matched by the regular expression \verb'[a-z_][a-zA-Z_]+'.
      Enumerating all the characters in a choice expression would be cumbersome,
      producing $(a+b+c+\dots+z+\_)(a+\dots+z+A+\dots+Z+\_)^+$. Therefore, \reml
      supports character classes as well as ranges and internally converts them
      to the longer choice expression.
      
      Inverted character classes such as \verb'[^cde]' match any symbol from
      $\Sigma$, except the ones after the caret. In this case, the language
      described by the expression would be $L(\Sigma) \setminus L((c+d+e))$.
      Such character classes are first converted to positive character classes,
      after which they are then converted to a choice using the above method.

   \item The wildcard character \verb'.' or in \ocamllex \verb'_' is implemented
      as a predefined full character class containing all of $\Sigma$. It is
      equivalent to \verb'[^]', the character class exempting no symbols.

   \item Backreferences are found in several regular expression dialects,
      including Perl and PCRE. In GNU \texttt{egrep}, the expression
      \verb'(a*)b\1' describes the language $S \to b | a S a$, which belongs to
      the class of context-free grammars and no longer to the class of regular
      grammars.

      A more complex example that does not even fall into the class of CFGs is
      \verb'([abc]*)b\1', which matches any sequence of $a \dots c$, followed by
      $b$, followed by the same sequence matched before the $b$. E.g. $abcbabc$
      could be matched by this expression.

      Since nondeterministic finite automata cannot recognise context-free
      grammars, let alone context-sensitive grammars, backreferences are not
      supported in this partial derivatives based approach.
      
\end{itemize}


\subsection{Syntax in software tools}

The abstract syntax and terminology presented here is generally not what is used
in software implementing regular expression matching. There are several syntaxes
in wide use that differ not only in accepted language constructs but also in
handling of meta-characters.

POSIX uses a different terminology to define what basic and extended regular
expressions are. POSIX Basic Regular Expressions (BRE) take a conservative
approach and define fewer meta-characters than Extended Regular Expressions
(ERE):

\begin{itemize}
   \item \verb'^' matches with zero width at the beginning of a line.
   \item \verb'$' matches with zero width at the end of a line.
   \item \verb'.' represents the wildcard character as discussed above.
   \item \verb'*' represents the Kleene star.
   \item Character classes such as \verb'[abc]' are recognised.
\end{itemize}

Matching these characters requires escaping them by a backslash. Other operators
also require the backslash character to be recognised as a meta-character:

\begin{itemize}
   \item \verb'\{n,m\}' with integers $n$ and $m$ matches the preceding
      expression at least $n$ and at most $m$ times, so that e.g.
      \verb'a\{1,2\}' matches $a$ or $aa$. The upper bound may be omitted, in
      which case the expression matches at least $n$ times with no restriction
      beyond that.
   \item \verb'\(r\)' defines a capturing group, which can be referenced by
      backreferences of the form \verb'\1' where any digit between 1 and 9 is a
      valid backreference.
   \item \verb'\+' is an extension accepted in some implementations that is
      equivalent to \verb'\{1,\}'.
   \item \verb'\?' is also an extension equivalent to \verb'\{0,1\}'. Both
      extensions are not part of the POSIX standard, but accepted by e.g. GNU
      \texttt{grep}.
\end{itemize}

The difference between Basic and Extended regular expressions in POSIX is that
the latter does not require backslashes for the above meta-characters, and
operators \verb'+' and \verb'?' are supported by the standard. Backslashes have
the opposite effect in ERE, in that they are used to escape the above
meta-characters so that they may be used in a literal match.

Furthermore, while the grouping operator in POSIX ERE does create an atomic
expression, which can be used as an argument to other operators, backreferences
are not supported in ERE.


\section{Other efficient sub-matching algorithms}

\subsection{Tagged NFA/DFA}

In his Master's Thesis \cite{laurikari}, Ville Laurikari describes a method of
constructing tagged nondeterministic finite automata (TNFA) and their conversion
to tagged deterministic finite automata (TDFA).


\subsection{Submatching with OBDDs}

HP.


\subsection{Bi-directional matching}

S. Haber et al.


\section{Regular expression matching engines}

This section is dedicated to describing and comparing various existing
implementations of regular expression matching and submatching. The criteria
used for comparison are:

\begin{itemize}
   \item Time and memory requirements for automaton construction.
   \item Matching performance, in speed and memory usage.
   \item Accepted regular expression syntax.
\end{itemize}

\section{RE2}

The RE2 engine by Russ Cox is a regular expression matcher based on finite
automata. In the first step, it constructs an NFA from the expression. When
matching an input string, it constructs a DFA on the fly and caches the states.
If the DFA cache exceeds a certain size, it is cleared and the DFA construction
starts over.

RE2 implements several low-level optimisations and exhibits high performance. It
provides specialised functions for specific matching tasks, e.g. a function that
only solves the word problem, that is, it tests whether the complete input is a
word in the language described by the regular expression. Another function may
test whether any substring of the input can be matched by the expression.

In addition, it performs several analyses over the regular expression that allow
it to choose specialised paths and algorithms for common cases of simple regular
expressions. For example, if the engine can always know which sub-match a
character is part of, it can avoid copying sub-match boundary sets. For example
in $(x:a^*)(y:b)$, all $a$s are assigned to the match variable $x$, and the
final $b$ is assigned to $y$, whereas in $(x:a^*)(y:a)$, after reading $aa$, it
can not yet be decided which variable the letters will be assigned to, resulting
in a boundary set with two elements: $\{\{(x,aa)\},\{(x,a),(y,a)\}\}$.

The problem solved by our pattern sub-matching with partial derivatives is the
most difficult one, and RE2 solves it by first running the DFA on the input to
find the end of the matched substring. After that, it runs another DFA in which
all concatenations of regular expressions are reversed, in order to find the
beginning of the matched substring. Finally, it runs an NFA to locate the
sub-matches.


\subsection{Automata construction}

The time taken by RE2 to compile a regular expression into an automaton is,
according to \cite{regexp3}, about three to four times more than by PCRE. The
resulting automaton object can become large, because of the cached DFA, but RE2
limits the size to a user-defined limit. By constructing DFA states on demand
rather than constructing the complete DFA up front, RE2 avoids the exponential
complexity of a full NFA state subset construction.


\subsection{Matching performance}

Using an automata based approach, RE2 avoids exponential time complexity for
regular expression matching, which a backtracking algorithm exposes for example
when matching $a^n$ with the expression $a?^na^n$. Russ Cox explains this
runtime behaviour in more detail in \cite{regexp1}. Common case optimisations
help RE2 outperform PCRE on large inputs, and the engine is not much slower than
PCRE on small strings.


\subsection{Accepted syntax}

RE2 accepts most common regular expression operators, but does not in any way
support negation or intersection of subexpressions. Back-references are not
supported in any automata based engine, and although RE2 contains a backtracking
implementation, it does not expose this in its API or syntax.


% vim:tw=80

\section{PCRE}


% vim:tw=80



% vim:tw=80
