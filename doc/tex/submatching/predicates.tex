\section{Language predicates}

In order to test whether a regular expression describes the empty language, we
define a boolean predicate $\isempty :: r \to \{0, 1\}$, which is an
approximation of $L(r) = \emptyset$. Since $\isempty(r) \neq \neg\isempty(\neg
r)$, we need to expand the negated case. We define a separate predicate
$\isemptyneg :: r \to \{0, 1\}$ for this case so we can easily prove termination
of the predicate.

\begin{defn}
   Approximation for $L(r) = \emptyset$.

   \begin{tabular}{lll}
      $\isempty(\emptyset)$	& $=$ & $1$ \\
      $\isempty(\varepsilon) = \isempty(p^*) = \isempty(l)$
         & $=$ & $0$ \\
      $\isempty(r^n)$		& $=$ & $\isempty(r)$ \\
      $\isempty(r_1 + r_2)$	& $=$ & $\isempty(r_1) \wedge \isempty(r_2)$ \\
      $\isempty((r_1, r_2))$	& $=$ & $\isempty(r_1) \vee \isempty(r_2)$ \\
      $\isempty(r_1 \cap r_2)$	& $=$ & $\isempty(r_1) \vee \isempty(r_2)$ \\
      $\isempty(\neg r)$		& $=$ & $\isemptyneg(r)$ \\
   \end{tabular}

   Specialisation for negation:

   \begin{tabular}{lll}
      $\isemptyneg(\varepsilon)$	& $=$ & $1$ \\ 
   \end{tabular}

   Although $\neg(\Sigma^*)$ describes the empty language, we do not filter it
   here, since this is an approximation.

   \begin{tabular}{lll}
      $\isemptyneg(r^*) = \isemptyneg(l) = \isemptyneg(\emptyset)$	& $=$ & $0$ \\
   \end{tabular}

   De Morgan's laws:

   \begin{tabular}{lll}
      $\isemptyneg(r_1 \cap r_2)$	& $=$ & $\isemptyneg(r_1) \wedge \isemptyneg(r_2)$ \\
      $\isemptyneg(r_1 + r_2)$	& $=$ & $\isemptyneg(r_1) \vee \isemptyneg(r_2)$ \\
      $\isemptyneg((r_1, r_2))$	& $=$ & $\isemptyneg(r_1) \vee \isemptyneg(r_2)$ \\
   \end{tabular}

   Double negation:

   \begin{tabular}{lll}
      $\isemptyneg(\neg r)$	& $=$ & $\isempty(r)$ \\
   \end{tabular}
\end{defn}


% vim:tw=80
