\chapter{Submatching with sets of pattern sets}
\label{submatching}

This chapter defines the syntax for extended regular expressions, on which the
semantics of a pattern match is then specified as a logical relation.  After the
theoretical behaviour of pattern matching is defined, we describe how a matching
automaton can be built by computing partial derivatives of patterns and regular
expressions. The partial derivatives computation used is based on a
representation of regular expressions as sets of expression sets.


\section{Extended regular expression submatching}
\label{ere-submatch}

The definition of partial derivatives from \cite{pdpat} has been combined with
the extended regular expression syntax as defined in \cite{pdere}. The regular
expression pattern syntax from Sulzmann and Lu is amended with additional
boolean operators for extended regular expressions.

\begin{defn}[Extended regular expression patterns]
   \label{defn-syn}
   Syntax:

   Words:

   \begin{tabular}{lrll}
      $w$	& $::=$	& $\varepsilon$		& Empty word	\\
		& $|$	& $l \in \Sigma$	& Letters	\\
		& $|$	& $lw$			& Concatenation	\\
   \end{tabular}

   Extended patterns:

   \begin{tabular}{lrll}
      $p$	& $::=$	& $(x:r)$			& Variables Base	\\
		& $|$	& $(x:p)$			& Variables Group	\\
		& $|$	& $p+p$				& Choice		\\
		& $|$	& $(p,p)$			& Concatenation		\\
		& $|$	& $p_e \cap p_e$		& Intersection		\\
        	& $|$	& $p_e^n$			& Repetition		\\
		& $|$	& $p^*$				& Kleene star		\\
   \end{tabular}
\end{defn}


Pattern and expression repetition in the form $r^n$ is essentially the same as
concatenating the same pattern or expression $n$ times. However, in a pattern,
this would cause variable binding clashes. For example, when expanding
$(x:a)^2$ to $(x:a)(x:a)$, the variable $a$ appears twice. Since we do not
allow the same variable name to appear in a pattern, the definition of partial
derivatives incorporates direct support for this form. Variable renaming could
solve this issue, but defining a pattern matching relation for the case of
repetition is the more straightforward solution.

The partial derivative of a negated pattern is not defined, because it cannot
be sensibly decided which part of a non-match should be assigned to which
pattern variable. The assignment is decidable if and only if every sub-pattern
always consumes the complete input, but for these patterns, their negation can
be pushed into the pattern. For variables base, $\neg(x:p) \equiv (x:\neg p)$,
and for pattern intersection, by De Morgan's law $\neg((x:p_1) \cap (y:p_2))
\equiv (x:\neg p_1) + (y:\neg p_2)$. Variable assignment can not be decided
for other patterns. E.g. when matching the string ``ccc'' with the pattern
$\neg((x:a)(y:b))$, it is impossible to decide which part of the string should
be assigned to $x$ and which to $y$. Therefore, we represent a subset of
pattern negations in terms of a negation normal form transformation that must
be applied before beginning the pattern match.

\needspace{6cm}

\begin{defn}[Negation normal form]
   \label{defn-negnorm}
   We define the following transformation:

   \begin{tabular}{lll}
      $\negnorm(\neg(x : r))$		& $=$	& $(x : \neg r)$					\\
      $\negnorm(\neg(x : p))$		& $=$	& $(x : \negnorm(\neg p)$				\\
      Double negation:			&	&							\\
      $\negnorm(\neg(\neg p))$		& $=$	& $\negnorm(p)$						\\
      De Morgan's laws:			&	&							\\
      $\negnorm(\neg(p_1 \cap p_2))$	& $=$	& $(\negnorm(\neg p_1) + \negnorm(\neg p_2)$		\\
      $\negnorm(\neg(p_1 + p_2))$	& $=$	& $(\negnorm(\neg p_1) \cap \negnorm(\neg p_2)$		\\
      Distributive laws:		&	&							\\
      $\negnorm(x : p)$			& $=$	& $(x : \negnorm p)$					\\
      $\negnorm(p_1 + p_2)$		& $=$	& $\negnorm(p_1) + \negnorm(p_2)$			\\
      $\negnorm(p_1 \cap p_2)$		& $=$	& $\negnorm(p_1) \cap \negnorm(p_2)$			\\
      $\negnorm(p_1, p_2)$		& $=$	& $(\negnorm(p_1), \negnorm(p_2))$			\\
      $\negnorm(p_1^*)$			& $=$	& $\negnorm(p_1)^*$					\\
      $\negnorm(p_1^n)$			& $=$	& $\negnorm(p_1)^n$					\\
      Trivial atomic pattern:		&	&							\\
      $\negnorm(x : r)$			& $=$	& $(x : r)$						\\
   \end{tabular}
\end{defn}

We do not define negation of patterns containing concatenation and iteration,
since there is no sensible way to assign pattern variables to submatches.

In the remainder of this paper, we will use the term ``regular expression''
for extended regular expressions, and refer to their instances as $r$ instead
of $r_e$.  Analogously, the term ``pattern'' is used for ``extended pattern''
and $p$ is an instance.

\begin{defn}[Matching environments and language of regular expressions]
   \label{defn-lang}
   Syntax and language definitions:

   Environments:

   \begin{tabular}{lrll}
      $\Gamma$	& $::=$	& $\{x:w\}$			& Variable binding	\\
		& $|$	& $\Gamma \uplus \Gamma$	& Ordered multi-set of
							  variable bindings	\\
   \end{tabular}

   \needspace{5cm}

   Language:

   \begin{tabular}{lll}
      $L(r_1 + r_2)$	& $=$	& $L(r_1) \cup L(r_2)$					\\
      $L((r_1, r_2))$	& $=$	& $\{ w_1w_2 \where w_1 \in L(r_1), w_2 \in L(r_2) \}$	\\
      $L(r_1 \cap r_2)$	& $=$	& $L(r_1) \cap L(r_2)$					\\
      $L(\neg r)$	& $=$	& $\Sigma^* \setminus L(r)$				\\
      $L(r^*)$		& $=$	&
         $\{\varepsilon\} \cup \{ w_1 \dots w_n \where i
         \in \{ 1, \dots, n \}, w_i \in L(r) \}$	\\
      $L(r^n)$		& $=$	&
         $\{ w_1 \dots w_n \where w_n \in L(r) \}$	\\
      $L(l)$		& $=$	& $\{l\}$						\\
      $L(\varepsilon)$	& $=$	& $\{\varepsilon\}$					\\
      $L(\phi)$		& $=$	& $\emptyset$						\\
   \end{tabular}
\end{defn}

% TODO: prove L(r^n)

The term $\Sigma^*$ is used as abbreviation for $\{ w \where w \in \Sigma^* \}$.

The variable environments are the same as in \cite{pdpat}, the language
definition has been extended with the definitions from \cite{pdere}.
Juxtaposition of terms is used interchangeably with the explicit pair syntax
for concatenation. This shortened syntax follows from the language definition
for term concatenation, which is the set product of the two languages of the
term operands.


\section{Pattern matching relation}
\label{patmatchrel}

We write the union of pattern match environments as $\uplus$, which is, since we
use ordered multi-sets to represent an environment, equivalent to list
concatenation. We formalise the pattern matching relation $w \vdash p \leadsto
\Gamma$ based on the definition in \cite{pdpat}, extending it with
intersection. We do not define a relation on pattern negation, since at this
point, these must have been put in the negation normal form using $\negnorm$.

\begin{defn}[Pattern matching relation]
   Defined as logical relations.

   \begin{minipage}{0.4\textwidth}
      \[\text{(VarBase)}\, \frac
         {w \in L(r)}
         {w \vdash x : r \leadsto \{x : w\}}
      \]
   \end{minipage}
   \begin{minipage}{0.5\textwidth}
      \[\text{(VarGroup)}\, \frac
         {w \vdash p \leadsto \Gamma}
         {w \vdash x : p \leadsto \{x : w\} \uplus \Gamma}
      \]
   \end{minipage}

   \begin{minipage}{0.4\textwidth}
      \[\text{(Concat)}\, \frac
         {\begin{array}{c}
            w = w_1 w_2 \\
            w_1 \vdash p_1 \leadsto \Gamma_1 \\
            w_2 \vdash p_2 \leadsto \Gamma_2
         \end{array}}     
         {w \vdash (p_1, p_2) \leadsto \Gamma_1 \uplus \Gamma_2}
      \]
   \end{minipage}
   \begin{minipage}{0.5\textwidth}
      \[\text{(Star)}\, \frac
         {\begin{array}{c}
            w = w_1 \dots w_n \\
            w_i \vdash p \leadsto \Gamma_i \text{ for } i = 1 \dots n
         \end{array}}     
         {w \vdash p^* \leadsto \Gamma_1 \uplus \dots \uplus \Gamma_n}
      \]
   \end{minipage}

   \begin{minipage}{0.4\textwidth}
      \[\text{(ChoiceL)}\, \frac
         {w \vdash p_1 \leadsto \Gamma_1}
         {w \vdash p_1 + p_2 \leadsto \Gamma_1}
      \]
   \end{minipage}
   \begin{minipage}{0.4\textwidth}
      \[\text{(ChoiceR)}\, \frac
         {w \vdash p_2 \leadsto \Gamma_2}
         {w \vdash p_1 + p_2 \leadsto \Gamma_2}
      \]
   \end{minipage}

   \[\text{(Intersect)}\, \frac
      {\begin{array}{c}
         w \vdash p_1 \leadsto \Gamma_1 \\
         w \vdash p_2 \leadsto \Gamma_2
      \end{array}}     
      {w \vdash p_1 \cap p_2 \leadsto \Gamma_1 \uplus \Gamma_2}
   \]

\end{defn}

As described in section \ref{ere-submatch}, the matching function yielded by the
pattern partial derivative can be customised with an $\iterate$. If this
function is defined as one that keeps all matches, the number of possible
matches can become exponential, if the pattern matching relation is defined as
above.

We consider the pattern $(x:A^*)$ and the input $A^n$. This pattern has one
possible match, which is the complete input $A^n$. If we now add an iterating
operator $p^*$ to the pattern, we match with the pattern $(x:A^*)^*$. The number
of possible matches is now $2^{n-1}$, namely all combinations of the set $\{
A^1, \dots, A^n \}$ that yield the original input $A^n$.

If the Kleene star is applied again, the number of matches does not change, but
the number of states the automaton ends with will be $m^{n-1}$ where $m$ is the
number of Kleene stars in tetration: $S((x:A^*)^{^m*}, A^n) = m^{n-1}$.

This exponential number of matches can be reduced if instead of an ordered
multi-set of matches, we use an unordered multi-set for the matching
environment. E.g. the match $\{ x:A; x:AA \}$ and $\{ x:AA; x:A \}$ will be
equivalent and counted only once. The resulting match-sets will be the set of
parts of all partitions of $n$ represented as $A^n$. Consequently, the number of
match-sets will be equal to the cardinality of the set of sets of parts of all
partitions of $n$. % This number is polynomial. TODO: really? and how many?

We can reduce the number of matches to one by defining a deterministic pattern
matching relation. The most commonly used disambiguation strategy used is the
greedy left-most matching strategy as used by Perl style regular expression
engines. A different strategy is used by POSIX, which favours the longest word
match. We employ the Perl strategy, which is natural for partial derivatives, as
we can simply take the first matched variable from the ordered multi-set that is
the matching environment.


\section{Partial derivatives}
\label{exprsets}

The syntax of patterns and regular expressions, as well as the formal
specification of the pattern matching relation can now be used to define the
pattern partial derivative function.

\subsubsection{Sets of pattern sets}

The partial derivative operations are defined on extended regular expressions
and result in sets of expression sets. Pattern partial derivatives as defined in
\cite{pdpat} result in a list of pairs containing the derivated term as well as
a transition function $x \mapsto l$ recording the fact that the letter $l$ has
been consumed by a pattern under a variable $x$. This function must be recorded
in the set of pattern sets, as well. Since the equivalent of a derivated term is
the expression set $\eset$, we define the set of pattern sets as: $\{ (\{ p \},
x \to l) \}$, the set of pairs containing an expression set and the associated
transition function. Sets of pattern sets are defined similarly to sets of
expression sets in definition \ref{defn-esets}.

\needspace{3cm}
\begin{defn}[Sets of pattern sets]
   \label{defn-psets}
   Syntax:

   \begin{tabular}{lll}
      $\pset$	& $::=$ & $(\{r_1, r_2, \dots, r_n\}, x \to l)$ \\
      $\psets$	& $::=$ & $\{\pset_1, \pset_2, \dots, \pset_n\}$ \\
   \end{tabular}
\end{defn}

Similarly to the additional requirement for Sulzmann and Lu's pattern partial
derivatives, our sets of pattern sets are in fact ordered sets, ordered by their
origin. That means the pattern $(x:ac)(y:ab)$ represented as set of pattern sets
is $\{\{x:ac\}, \{y:ab\}\}$, exactly in that order. This post-condition is
required to ensure that matching strategies such as the greedy left-most and
POSIX match operate correctly.


\subsubsection{Conversions}

The operators $\esetOfRe$ and $\reOfEset$ can be defined analogously for
patterns to construct pattern sets and reconstruct the pattern from them.

\begin{defn}[Patterns to pattern sets]
   \label{defn-re2pset}
   Conversions from patterns.

   Pattern set from expression pattern:
   $\cdot\esetOfRe :: p \to \pset$

   \begin{tabular}{lll}
      $(p_1 \cap p_2)\esetOfRe$	& $=$	& $p_1\esetOfRe \cup p_2\esetOfRe$	\\
      $p\esetOfRe$			& $=$	& $\{p\}$			\\
   \end{tabular}
\end{defn}

It should be noted that in order to break a pattern or expression into a set of
pattern or expression sets, the highest level operator must be choice or
intersection. In particular, a pattern group $(x:p)$ at the highest level of the
syntax tree makes it impossible to break a pattern into pieces.

For a pattern group containing a choice operator, it would be possible to split
it into multiple pattern groups with the same variable name, each containing one
choice. However, this would break the invariant that every variable name exists
exactly once in the pattern, so splitting would require variable renaming and
tracking.

The inverse operator is defined analogously to definition \ref{defn-re2esets}.

\begin{defn}[Pattern sets to patterns]
   \label{defn-re2psets}
   Conversions from pattern sets back to patterns.

   Pattern from pattern set:
   $\cdot\reOfEset :: \pset \to r$

   \begin{tabular}{lll}
      $\{p\}\reOfEset$			& $=$	& $p$				\\
      $(\{p\} \cup \pset)\reOfEset$	& $=$	& $p \cap \pset\reOfEset$	\\
   \end{tabular}
\end{defn}

We do not define $\esetsOfRe$ and $\reOfEsets$ for sets of pattern sets, and
there is no need for them, but the final pattern derivative as set of patterns
with transition functions can be obtained from a pattern using the function
$\cdot \setminus_p \cdot$ detailed and illustrated in section \ref{nfa}.


\subsubsection{Operators on sets of pattern sets}

The expression set operators from definition \ref{defn-eset-ops} must be defined
for pattern derivatives, but since pattern derivatives carry a transition
function $f$, these must be sensibly combined in the set operators.

\begin{defn}[Distributive laws on sets of pattern sets]
   \label{defn-pset-ops}
   The normal distributive laws can similarly to \ref{defn-eset-ops} be extended
   to sets of pattern sets.

   \begin{tabular}{lll}
      $\psets \circledcdot_{mod} p_2$
         & $=$
         & $\{ (\{ (p_1, p_2) \where p_1 \in \eset \}, f \circ mod) \where (\eset, f) \in \psets \}$
         \\

      $\psets_1 \circledcap \psets_2$
         & $=$
         & $\{
              (\eset_1 \cup \eset_2, f_1 \circ f_2)
              \where (\eset_1, f_1) \in \psets_1, (\eset_2, f_2) \in \psets_2
           \}$
         \\
   \end{tabular}
\end{defn}

In $\circledcdot$, the $mod$ function can modify the environment before passing
it to the transition function.  The operator $\circledneg$ is not defined for
sets of pattern sets, since there is no negation for patterns.


\subsection{Pattern partial derivatives}

We define the partial derivative of an extended pattern expression using the
operators $\circledcdot$ and $\circledcap$, omitting negation. This definition
is an extension of definition \ref{defn-pd-eset} with the semantics from the
pattern matching relation in definition \ref{patmatchrel}.

\begin{defn}[Pattern partial derivative]
   \label{defn-pd-pset}
   Pattern partial derivatives as sets of pattern sets.

   \[\nrdef{1p} \dda(x:r)
      = \{ (\{ \eset\reOfEset \}, \updatexa) \where \eset \in \dda(r) \}\]
   \[\nrdef{2p} \dda(x:p)
      = \{ (\{ (x:\pset\reOfEset) \}, \updatexa \circ f) \where (\pset, f) \in \dda(p) \}\]
   \[\nrdef{3p} \dda(p_1 + p_2) = \dda(p_1) \cup \dda(p_2)\]
   \[\nrdef{4p} \dda(p^*) = \dda(p) \circledcdot_{\iterate} p^*\]
   \[\nrdef{5p} \dda((p_1, p_2)) =
       \begin{cases}
          \dda(r_1) \circledcdot_{id} p_2 & \text{if } \varepsilon \not\in L(p_1\reOfEsets) \\
          \dda(r_1) \circledcdot_{id} p_2 \cup \dda(p_2) & \text{otherwise}
       \end{cases}
   \]
   \[\nrdef{6p} \dda(p_1 \cap p_2) = \dda(p_1) \circledcap \dda(p_2)\]
   \[\nrdef{7p} \dda(p^n) =
       \begin{cases}
          \{ (\pset, f \circ \iterate) \where (\pset, f) \in \dda(p) \} & \text{if } n = 1 \\
          \dda(p) \circledcdot_{\iterate} p^{n-1} & \text{otherwise}
       \end{cases}
   \]
\end{defn}

In \nr{5p}, the identity function is given as modifier to the $\circledcdot$
operator, keeping the environment as is before passing it to the transition
function. The $\iterate$ function can modify the environment for the two
iterating patterns $p^*$ and $p^n$. It can choose to keep only the last match or
all matches, by renaming past recorded match variables. This function is
described in more detail in section 5.1 of \cite{pdpat}. The $\fv(p)$ function
finds all free variables used in the pattern, and is defined in section 3.1 of
the same paper.


\section{The submatch automaton}
\label{nfa}

Given the definition of the partial derivative of a submatching pattern $\dda$,
which yields a set of expression sets, we can define a function yielding the
partial derivative for a pattern $p$: $\cdot \setminus_p \cdot :: p \to l \to
\{(p, x \to l)\}$

\begin{defn}[Pattern partial derivative as set of tagged transitions]
   \[p \setminus_p l = \{ (\mathcal{P}\downarrow, f) \where (\mathcal{P}, f) \in \dd{l}(p) \}\]
\end{defn}

Using this function, the submatching automaton can be constructed by
iteratively constructing the partial derivative of the input pattern over each
letter in the alphabet. The resulting set of derivated terms is then filtered to
remove all those that describe the empty language. This filtered set is then
used as input and each term is derived over each letter again. This is done
until no new terms appear or all terms yield the empty language, i.e. $L(p) =
\emptyset$.

\begin{eg}
   Given the alphabet $\Sigma = \{a, b, c\}$ and the pattern expression $p =
   (x:a, p_2)$ where $p_2 = (y:\neg c, p_3)$ and $p_3 = (z:ab + ac)$, we can
   construct the automaton as follows:

   \begin{tabular}{lll}
      $\dda(a)$
      & $=_{\nr{2}}$ &
         $\{\{ \varepsilon \}\}$
   \end{tabular}

   \begin{tabular}{lll}
      $\dda(((x:a), p_2))$
      & $=_{\nr{5p}}$ &
         $\dda(x:a) \circledcdot_{\id} p_2$
      \\ & $=_{\nr{1p}}$ &
         $\{ (\{ \eset\reOfEset \}, \updatexa) \where \eset \in \dda(a) \}
            \circledcdot_{\id} p_2$
      \\ & $=$ &
         $\{ (\{ \varepsilon \}, \updatexa) \}
            \circledcdot_{\id} p_2$
      \\ \multicolumn{3}{l}
         {By definition \ref{defn-pset-ops}, and we cancel the identity function
         in the update chain.}
      \\ & $=$ &
         $\{ (\{ (\varepsilon, p_2) \}, \updatexa) \}$
      \\ \multicolumn{3}{l}
         {Simplification $(\varepsilon, p) = p$:}
      \\ & $=$ &
         $\{ (\{ p_2 \}, \updatexa) \}$
   \end{tabular}

   The resulting set of pattern sets contains the partial derivative over the
   letter $a$. The partial derivative over $b$ and $c$ will yield the empty
   language, resulting in the set $\big\{ (\{ (\phi, p_2) \}, \updatexa)
   \big\}$. These are filtered out at the end of the iteration. The remaining
   sets are converted to partial derivatives according to the $\cdot \setminus_p
   \cdot$ operator, resulting in $p \setminus_p a = \{ (p_2, \updatexa) \}$. In
   the next iteration, we continue with the set of derivated terms from the
   previous iteration, in this case just $p_2$.

   \begin{tabular}{lll}
      $\dda(\neg c)$
      & $=_{\nr{8}}$ &
         $\circledneg (\dda(c))$
      \\ & $=_{\nr{1}}$ &
         $\circledneg (\{\{\phi\}\})$
      \\ \multicolumn{3}{l}
         {By definition \ref{defn-eset-ops}:}
      \\ & $=$ &
         $\{\{\neg\phi\}\}$
   \end{tabular}

   \needspace{5cm}

   \begin{tabular}{lll}
      $\dda(((y:\neg c), p_3))$
      & $=_{\nr{5p}}$ &
         $\dda(y:\neg c) \circledcdot_{\id} p_3 \cup \dda(p_3)$
      \\ & $=_{\nr{1p}}$ &
         $\{ (\{ \eset\reOfEset \}, \update{y}{a}) \where \eset \in \dda(\neg c) \}
            \circledcdot_{\id} p_3 \cup \dda(p_3)$
      \\ & $=$ &
         $\{ (\{ \neg\phi \}, \update{y}{a}) \}
            \circledcdot_{\id} p_3 \cup \dda(p_3)$
      \\ & $=$ &
         $\{ (\{ (\neg\phi, p_3) \}, \update{y}{a}) \} \cup \dda(p_3)$
   \end{tabular}

   Since $\varepsilon \in L(\neg c)$, we also need to compute $\dda(p_3)$ and
   unite the resulting set of pattern sets with $\dda(y:\neg c)$.

   \begin{tabular}{lll}
      $\dda(ab + ac)$
      & $=_{\nr{3}}$ &
         $\dda(ab) \cup \dda(ac)$
      \\ & $=_{\nr{5}}$ &
         $\dda(a) \circledcdot b \cup \dda(a) \circledcdot c$
      \\ & $=_{\nr{2}}$ &
         $\{\{\varepsilon\}\} \circledcdot b \cup \{\{\varepsilon\}\} \circledcdot c$
      \\ & $=$ &
         $\{\{\varepsilon b\}\} \cup \{\{\varepsilon c\}\}$
      \\ & $=$ &
         $\{\{\varepsilon b\}, \{\varepsilon c\}\}$
   \end{tabular}

   \begin{tabular}{lll}
      $\dda(z:ab + ac)$
      & $=_{\nr{1p}}$ &
         $\{ (\{ \eset\reOfEset \}, \update{z}{a}) \where \eset \in \dda(ab + ac) \}$
      \\ & $=$ &
         $\{ (\{ \eset\reOfEset \}, \update{z}{a}) \where \eset \in \{\{\varepsilon b\}, \{\varepsilon c\}\} \}$
      \\ & $=$ &
         $\{ (\{ \varepsilon b \}, \update{z}{a}), (\{ \varepsilon c \}, \update{z}{a}) \}$
      \\ \multicolumn{3}{l}
         {Simplification $(\varepsilon, p) = p$:}
      \\ & $=$ &
         $\{ (\{ b \}, \update{z}{a}), (\{ c \}, \update{z}{a}) \}$
   \end{tabular}

   The resulting set of pattern sets is
   \begin{align*}
     \dda(\neg c) \cup \dda(z:ab + ac) =
     \big\{
     & (\{ (\neg\phi, p_3) \}, \update{y}{a}), \\
     & (\{ b \}, \update{z}{a}), \\
     & (\{ c \}, \update{z}{a})
     \big\}
   \end{align*}

   After using the same method to iteratively construct the entire automaton,
   the states and transitions will be as in the graph in figure \ref{img-nfa}.

   \img{nfa}{15cm}{Tagged transitions in NFA}

\end{eg}


\section{Language predicates}

As an optimisation, we detect whether it is worth computing the partial
derivative of an expression by testing whether the expression accepts any
language, at all. If it does not, we can save the time spent on examining that
path. E.g. for the expression $a^n \cap a^*$, the NFA construction yields $2n +
1$ states, for $(a+b)^n \cap (a+b)^*$ it yields $4n + 1$ states. With the empty
language filter, the automaton for both expression only contains $n + 1$ states.

In order to test whether a regular expression describes the empty language, we
define a boolean predicate $\isempty :: r \to \{0, 1\}$, which is an
approximation of $L(r) = \emptyset$. An exact solution to this problem is, for
negation, equivalent to testing whether the language under negation accepts
$\Sigma^*$. As shown by \cite{stoc73}, this problem is PSPACE-complete. The
approximation described here requires linear time and constant space.

Since $\isempty(r) \neq \neg\isempty(\neg r)$, we need to expand the negated
case. We define a separate predicate $\isemptyneg :: r \to \{0, 1\}$ for this
case so we can trivially prove termination of the predicate, since the argument
to the recursive call is a strict subset of the function argument.

In other words, if a language is non-empty, one can not directly derive that its
negation is empty. For example, considering the simple atomic regular expression
$r = a$ matching a single letter from the alphabet, the language of the
negation, $L(\neg a)$ is in fact $\Sigma^* \setminus a$. Conversely, however,
the negation of the empty language, regardless of its form as regular
expression, is always $\Sigma^*$, so that e.g. $L(\neg(a \cap b)) =
L(\neg(\phi)) = \Sigma^* \setminus \phi = \Sigma^*$.

This function is an approximation in that if it yields \true, then $L(r) =
\emptyset$, but if it yields \false, $L(r) \neq \emptyset$ may not follow. For
instance, $\isempty(a \cap b) = 0$, but $L(a \cap b) = \emptyset$.

\begin{defn}[Approximation for $L(r) = \emptyset$]
   \label{defn-isempty}
   $\isempty(\phi)$ is true if and only if $L(r) = \emptyset$.

   \begin{tabular}{lll}
      $\isempty(\phi)$	& 	$=$ & $1$ \\
      $\isempty(\varepsilon) = \isempty(r^*) = \isempty(l)$
         & $=$ & $0$ \\
      $\isempty(r^n)$		& $=$ & $\isempty(r)$ \\
      $\isempty(r_1 + r_2)$	& $=$ & $\isempty(r_1) \wedge \isempty(r_2)$ \\
      $\isempty((r_1, r_2))$	& $=$ & $\isempty(r_1) \vee \isempty(r_2)$ \\
      $\isempty(r_1 \cap r_2)$	& $=$ & $\isempty(r_1) \vee \isempty(r_2)$ \\
      $\isempty(\neg r)$	& $=$ & $\isemptyneg(r)$ \\
   \end{tabular}

   A specialisation of $\isempty$ for negation is $\isemptyneg(r)$, which has
   the meaning of $\isempty(\neg r)$. Note that although $\neg(\Sigma^*)$
   describes the empty language, we do not filter it here, since this is an
   approximation.

   \begin{tabular}{lll}
      $\isemptyneg(\varepsilon) = \isemptyneg(r^*) = \isemptyneg(l) = \isemptyneg(\phi)$ & $=$ & $0$ \\
   \end{tabular}

   \begin{tabular}{lll}
      $\isemptyneg((r_1, r_2))$		& $=$ & $\isemptyneg(r_1)$ \\
   \end{tabular}

   De Morgan's laws:

   \begin{tabular}{lll}
      $\isemptyneg(r_1 \cap r_2)$	& $=$ & $\isemptyneg(r_1) \wedge \isemptyneg(r_2)$ \\
      $\isemptyneg(r_1 + r_2)$		& $=$ & $\isemptyneg(r_1) \vee \isemptyneg(r_2)$ \\
   \end{tabular}

   Double negation:

   \begin{tabular}{lll}
      $\isemptyneg(\neg r)$	& $=$ & $\isempty(r)$ \\
   \end{tabular}
\end{defn}

We provide a partial proof for the above definition by showing that
$\isempty(r)$ iff $L(r) = \emptyset$. The assertion that conversely, $L(r) =
\emptyset$ iff $\neg\isempty(r)$ and therefore $\neg\isempty(r) \Rightarrow L(r)
\neq \emptyset$ is not required for the approximation, since the function is
only necessary to rule out false positives and therefore the possibility of
removing valid transitions from the resulting automaton.

\begin{lem}[Correctness of $\isempty$ as an approximation]
   $\isempty(r) \Rightarrow L(r) = \emptyset$

   \proof
   It must be shown that for every instance of $r$ in which $\isempty(r)$ is
   \true, it holds that $L(r) = \emptyset$. We assume that $\Sigma$ is not empty,
   so that $\Sigma^* \setminus \{\varepsilon\} \neq \emptyset$. The premise for
   each part of the proof is that $L(r) = \emptyset$. If it is not, then
   $\isempty(r)$ is not \true.

   \begin{enumerate}[label=\textbf{(\arabic*)}]
      \item
         $L(\phi) = \emptyset \Leftarrow \isempty(\phi)$

      \item
         The cases for $L(\varepsilon)$, $L(r^*)$ and $L(l)$, as well as their
         negations and $L(\neg \phi)$ are omitted, since although their
         correctness can be proven, it is irrelevant, as we are only interested
         in cases yielding \true.

      \item
         If one of the intersecting languages is empty, then the intersection is
         also empty. \\
         $L(r_1 \cap r_2) = \emptyset$ \\
         $\Leftrightarrow L(r_1) \cap L(r_2) = \emptyset$ \\
         $\Leftrightarrow L(r_1) = \emptyset \vee L(r_2) = \emptyset$ \\
         $\Leftarrow \isempty(r_1) \vee \isempty(r_2)$ \\
         $\Leftrightarrow \isempty(r_1 \cap r_2)$

      \item
         If one of the choices is not empty, then the expression is not empty,
         thus both choices must be empty for the expression to be considered
         empty. \\
         $L(r_1 + r_2) = \emptyset$ \\
         $\Leftrightarrow L(r_1) \cup L(r_2) = \emptyset$ \\
         $\Leftrightarrow L(r_1) = \emptyset \wedge L(r_2) = \emptyset$ \\
         $\Leftarrow \isempty(r_1) \wedge \isempty(r_2)$ \\
         $\Leftrightarrow \isempty(r_1 + r_2)$

      \item
         If the left expression does not recognise any language, then the
         complete expression does not. If it does, but the right expression does
         not, then after matching the left expression, the complete expression
         runs into an error state, therefore not recognising any language. \\
         $L((r_1, r_2)) = \emptyset$ \\
         $\Leftrightarrow \{ w_1w_2 \where w_1 \in L(r_1), w_2 \in L(r_2) \} = \emptyset$ \\
         $\Leftrightarrow L(r_1) = \emptyset \vee L(r_2) = \emptyset$ \\
         $\Leftarrow \isempty(r_1) \vee \isempty(r_2)$ \\
         $\Leftrightarrow \isempty(r_1 \cap r_2)$

      \item
         If $r$ recognises a language, then any number of consecutive $r$ also
         does so. Conversely, if $L(r)$ is the empty language, then no number of
         consecutive $r$ will fix that. \\
         $L(r^n) = \emptyset$ \\
         $\Leftrightarrow \{ w_1 \dots w_n \where w_i \in L(r) \} = \emptyset$ \\
         $\Leftrightarrow L(r) = \emptyset$ \\
         $\Leftarrow \isempty(r)$ \\
         $\Leftrightarrow \isempty(r^n)$

      \item
         $L(\neg(r_1 \cap r_2)) = \emptyset$ \\
         De Morgan's law: \\
         $\Leftrightarrow L(\neg r_1 + \neg r_2) = \emptyset$ \\
         $\Leftrightarrow L(\neg r_1) \cup L(\neg r_2) = \emptyset$ \\
         $\Leftrightarrow L(\neg r_1) = \emptyset \wedge L(\neg r_2) = \emptyset$ \\
         $\Leftarrow \isempty(\neg r_1) \wedge \isempty(\neg r_2)$ \\
         $\Leftrightarrow \isemptyneg(r_1) \wedge \isemptyneg(r_2)$ \\
         $\Leftrightarrow \isemptyneg(r_1 \cap r_2)$

      \item
         $L(\neg(r_1 + r_2)) = \emptyset$ \\
         De Morgan's law: \\
         $\Leftrightarrow L(\neg r_1 \cap \neg r_2) = \emptyset$ \\
         $\Leftrightarrow L(\neg r_1) \cap L(\neg r_2) = \emptyset$ \\
         $\Leftrightarrow L(\neg r_1) = \emptyset \vee L(\neg r_2) = \emptyset$ \\
         $\Leftarrow \isempty(\neg r_1) \vee \isempty(\neg r_2)$ \\
         $\Leftrightarrow \isemptyneg(r_1) \vee \isemptyneg(r_2)$ \\
         $\Leftrightarrow \isemptyneg(r_1 + r_2)$

      \item
         $L(\neg(r_1, r_2)) = \emptyset$ is true if $L((r_1, r_2)) =
         L(\Sigma^*)$, which is achieved if $L(r_1) = L(\Sigma^*)$ or $L(r_1) =
         L(\Sigma^n)$ and $L(r_2)= L(\Sigma^*)$. Following from the definition
         of negation, these are equivalent to saying $L(\neg r_1) = \emptyset$
         or $L(\neg r_1) = L(\Sigma^*) \setminus L(\Sigma^n)$ and $L(\neg r_2) =
         \emptyset$. The latter cannot be used to decide the empty language
         problem in linear time, so we can rewrite the problem as $L(\neg(r_1,
         r_2)) = \emptyset \Leftarrow L(\neg r_1) = \emptyset \Leftarrow
         \isemptyneg(r_1)$.

      \item
         $L(\neg\neg r) = L(r)$, thus $\isempty(\neg\neg r) = \isempty(r)$.

   \end{enumerate}

   \qed
\end{lem}


In the derivation of $(r,r)$, i.e. expression or pattern concatenation, we need
to know whether the first expression may derive $\varepsilon$, the empty word.
In this case, we consider the second expression for the derivative, as well. For
the same reason we have a predicate for $L(r) = \emptyset$, we create a function
for $\varepsilon \in L(r)$. The two predicates have the same time and space
complexity. Unlike the above predicate, however, the $\nullable$ function is
exact. Its termination can be proved in the same way as above. Note that the
$\nullable$ predicate is merely a special case of the word problem.

\begin{defn}[Nullability]
   \label{defn-nullable}
   Exact function for $\varepsilon \in L(r)$.

   \begin{tabular}{lll}
      $\nullable(\phi)$		& $=$ & $0$ \\
      $\nullable(\varepsilon)$	& $=$ & $1$ \\
      $\nullable(r^*)$		& $=$ & $1$ \\
      $\nullable(r^n)$		& $=$ & $\nullable(r)$ \\
      $\nullable(r_1 + r_2)$	& $=$ & $\nullable(r_1) \vee \nullable(r_2)$ \\
      $\nullable((r_1, r_2))$	& $=$ & $\nullable(r_1) \wedge \nullable(r_2)$ \\
      $\nullable(r_1 \cap r_2)$	& $=$ & $\nullable(r_1) \wedge \nullable(r_2)$ \\
      $\nullable(\neg r)$	& $=$ & $\neg\nullable(r)$ \\
   \end{tabular}

\end{defn}

Unlike the predicate from definition \ref{defn-isempty}, the $\nullable$
predicate does not require a specialisation for $\neg r$, since the language $L$
of a negated regular expression $r$ is, as defined in \ref{defn-lang}, $\Sigma^*
\setminus L(r)$, i.e. every word, except the ones in $L(r)$. Therefore, the
question whether a symbol or word is in a language is equivalent to the question
whether it is not in the negated language.


\section{Declarative asynchronous lexers}
\label{lexer}

As mentioned in the introduction, submatching can be used to declaratively
specify a lexer for a programming language. We can extract the greedy left-most
match from the matching environment $\Gamma$ and pass the lexeme information to
the semantic action code.

One inelegance in the way most lexer functions are written is that they are
synchronous by design. Characters are read from the input stream until a
complete lexeme is read and no more characters can be accepted, after which the
stream position is reset to the end of the valid lexeme. The function returns
the value of the semantic action with the stream ready for a subsequent lexer
function call.

Instead of reading a single token at a time and resetting the stream, we can
define the combined lexer for all lexemes $r_1 \dots r_n$ as $(r_1 + \dots
r_n)^*$. The key idea here is to have an iterating expression for all lexemes.
Without any semantic actions, this combined pattern will match a complete valid
source text. Semantic actions may operate asynchronously, sending complete
matches to a channel from which a client, such as a parser for a context free
grammar, can wait, enabling concurrent lexing and parsing.


\subsection{Matching functions}

We extend on the idea of named submatches and instead of associating a name in
the matching environment with a submatch, we associate a function with each
pattern. Thus, instead of a pattern $(x:r)$ where variable $x$ is associated
with regular expression $r$, we write the pattern $(f:r)$ in which a function is
called when the sub-expression is successful.

Now, we need to define what it means for a pattern to be successful. Given for
instance the pattern $(f_1:A)(f_2:B)+(f_3:AC)$ and the input $AC$, after reading
an $A$, the sub-pattern $(f_1:A)$ has been matched, but after the next
character, $C$, the containing pattern fails, so that $f_1$ may not be called.

\subsubsection{Delayed actions}

It is clear that semantic action invocations must be delayed until it is certain
that the pattern they appeared in succeeds. In a lexer specification $((f_1:r_1)
+ \dots + (f_n:r_n))^*$, the actions should not be delayed until the complete
input has been consumed, since the input stream may be infinite or very large
and the storage requirements for delayed actions would be linear in the input
length. Therefore, the actions should only be delayed for as little time as
possible.

\needspace{5cm}
\begin{defn}[Match completeness]
   This defines when a match is considered complete.

   Supporting definitions:
   \begin{enumerate}

      \item A toplevel pattern as one that is not enclosed in any other patterns.
         
         E.g. in $(f_1:A+(f_2:B))+(f_3:C)$, the sub-patterns associated with
         $f_1$ and $f_3$ are toplevel patterns, while the one for $f_2$ is not,
         since it is enclosed in the pattern associated with $f_1$.

      \item Two patterns $p_1$ and $p_2$ are parallel when they consume the same
         input.

         E.g. in $(f_1:ABC)+(f_2:(f_3:A+B)^*C)$, while consuming $AB$, patterns
         associated with $f_1$, $f_2$ and $f_3$ are parallel. After consuming
         $C$, patterns for $f_1$ and $f_2$ are parallel, but the pattern for
         $f_3$ is no longer consuming any input.

   \end{enumerate}

   A pattern match $p_1$ is complete when
   \begin{enumerate}

      \item any enclosing pattern match, if such exists, is complete,

      \item it can no longer consume any more input,

      \item there is no incomplete pattern $p_2$ where $p_1$ and $p_2$ are
         parallel.

   \end{enumerate}

\end{defn}

\subsubsection{Longest match strategy}

A single pattern $p$ under the greedy left-most matching strategy will consume
as much of the input as possible. However, when using the above-mentioned
declarative lexer using $p' = p^*$, this is no longer true. Given, for instance,
a language with lexemes $A$, $B$, and $AB$, the lexer pattern would be $p =
(f_1:A) + (f_2:B) + (f_3:AB)$. Tokenising the input $AB$ would, under the greedy
left-most strategy, invoke the semantic action $f_3$.

With the pattern written as $p' = ((f_1:A) + (f_2:B) + (f_3:AB))^*$, the greedy
left-most match would favour the shorter matches, and an input of $AB$ would
first complete the match associated with $f_1$ with $A$ and then $f_2$ with $B$.

In order to maintain the matching behaviour of lexers, the toplevel choice over
all lexemes must follow a longest match policy rather than the greedy left-most
match used inside lexeme expressions.


% vim:tw=80
