\section{Submatch extraction}
\label{submatch}

As described in section \ref{ere-submatch}, the matching function yielded by
the pattern partial derivative can be customised with an $iterate_{fv(p)}$. If
this function is defined as one that keeps all matches, the number of possible
matches can become exponential.

We consider the pattern $(x:a^*)$ and the input $a^n$. This pattern has one
possible match, which is the complete input $a^n$. If we now add an iterating
operator $p^*$ to the pattern, we match with the pattern $(x:a^*)^*$. The
number of possible matches is now $2^{n-1}$, namely all combinations of the
set $\{ a^1, \dots, a^n \}$ that yield the original input $a^n$.

If the Kleene star is applied again, the number of matches does not change,
but the number of states the automaton ends with will be $m^{n-1}$ where $m$
is the number of kleene stars in tetration: $S((x:a^*)^{^m*}, a^n) = m^{n-1}$.

This exponential number of matches can be reduced if instead of an ordered
multi-set of matches, we use an unordered multi-set for the matching
environment. E.g. the match $\{ x:a; x:aa \}$ and $\{ x:aa; x:a \}$ will be
equivalent and counted only once. The resulting match-sets will be the set of
parts of all partitions of $n$ represented as $a^n$. Consequently, the number
of match-sets will be equal to the cardinality of the set of sets of parts of
all partitions of $n$. This number is polynomial. TODO: really? and how many?


% vim:tw=78
