\section{Partial derivatives}

In order to reduce the time spent in constructing the matching automaton,
Valentin M. Antimirov developed the concept of partial derivatives of regular
expressions in \cite{antimirov}. Instead of a single derivated term for each
input symbol, partial derivatives may produce an arbitrary number of terms.
Every time the derivative operation is performed for an choice expression, the
possible choices are split as an set of derivated terms.

Antimirov proved that the set of partial derivatives for any given expression is
finite. More importantly, he asserts that the number of partial derivatives of a
regular expression and recursively of all its derivated terms is at most one
more than the number of occurrences of symbols from the input alphabet. In other
words, the number of states in the non-deterministic matching automaton
constructed from partial derivatives is linear in the length of the initial
expression.

Consequently, an automaton construction algorithm based on Antimirov's ideas can
run in linear time over the regular expression size, and the actual NFA
simulation still runs in linear time over the input size, if all branches are
investigated simultaneously. The resulting matcher takes less memory but
slightly more time, since the sets of NFA states the automaton is currently in
must be computed on the fly. Since these sets are by themselves DFA states, they
may be cached in order to lazily perform a subset construction, resulting in a
deterministic finite automaton.


% vim:tw=80
