\section{PCRE}

The Perl Compatible Regular Expressions library is a C library that attempts to
bring the power of Perl regular expressions to other languages. PCRE and Perl
differ in a few instances, but their semantics and syntax are mostly equivalent.
In particular, Perl allows arbitrary Perl code to be executed within a match,
whereas PCRE does not include a Perl interpreter and can therefore never
implement this feature. However, PCRE does provide a way to call C code in named
``callout'' points.

\subsection{Automata construction}

PCRE is very fast at compiling a regular expression. The expression is compiled
to 


\subsection{Matching performance}

The default matching algorithm used by the library is based on backtracking. The
regular expression is translated to a program that operates by processing one
path of an expression at a time, going back to the last matched text and program
position.

For example, consider the regular expression $(x_1:a^*)(x_2:a)(x_3:b^*)(x_4:b)$,
which matches any number of $a$ followed by any number of $b$, but at least one
of each. Matching the string $aabb$ against this pattern with a backtracking
algorithm will first attempt process the sub-expression $a^*$ by repeatedly
processing the sub-expression $a$ until it fails, so that $x_1 = aa$. Then, it
tries to match $(x_2:a)$, which fails, because the input is now at $b$, so the
algorithm backs up to the position at which the last successful match began, and
proceeds to process the input $abb$.

This algorithm requires the matcher to process the input up to $2^n$ times, as
there might be an exponential number of paths to test.

PCRE offers a DFA implementation as alternative matching algorithm, which is not
Perl compatible, but very efficient. Furthermore, it includes a Just-In-Time
compiler based on sljit, the stackless JIT compiler, increasing matching
performance by several times\footnote{\texttt{pcregrep --no-jit} was between two
and five times slower than without \texttt{--no-jit} for various patterns.}.

A pattern that causes particularly bad performance in PCRE is $(a?)^na^*b$. For
this expression, PCRE takes exponential time, and a DFA takes linear time.


\subsection{Accepted syntax}

PCRE supports the richest syntax of all investigated regular expression
implementations. It supports the full syntax as specified by the Perl regular
expression man page.


% vim:tw=80
