\chapter{Conclusion}
\label{conclusion}

In this master's thesis, we combined the theory of pattern partial derivatives
from \cite{pdpat} with the sets of expression sets extension of partial
derivatives from \cite{pdere} to support extended regular expression
submatching. We have implemented a prototype named \dreml{} in OCaml so that its
performance could be compared to \mlulex, another existing program written in
SML.

Our timing results indicate that despite the increased complexity of computing
and operating on expression sets rather than computing a single derivative for
each expression, we can achieve competitive performance for automata
construction as well as at matching time.

Similarities and differences between our method and existing regular expression
engines supporting submatching have been shown. We have evaluated several
algorithms and shown how our algorithm improves upon them.

Future efforts will be directed at improving usability and performance of the
tool, so that it can be used as a drop-in replacement for \ocamllex{} with
additional functionality, while retaining or improving upon its performance.

Some ideas for future development:

\begin{itemize}

   \item Add Unicode support, building on the ideas implemented in \mlulex{} and
      presented in \cite{re-deriv}. This would improve compile time performance
      even for non-Unicode patterns.

   \item Perform static analysis on regular expressions and the resulting
      automaton to provide better error messages, both at compile time and at
      matching time.

   \item Provide an option to determinise the NFA and minimise the resulting
      DFA, at the expense of increased compile time.

   \item Implement an ML code generator producing mutually recursive functions
      in addition to the current table-based backend.

\end{itemize}

An OCaml implementation of the algorithms presented in this thesis can be
downloaded as source code from \url{https://github.com/pippijn/dreml}.


% vim:tw=80
