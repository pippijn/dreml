\chapter{Conclusion}
\label{conclusion}

In this master's thesis, we combined the theory of pattern partial derivatives
from \cite{pdpat} with the sets of expression sets extension of partial
derivatives from \cite{pdere} to support extended regular expression
submatching. We have implemented a prototype named \dreml{} in OCaml so that its
performance could be compared to \mlulex, another existing program written in
SML.

Our timing results indicate that despite the increased complexity of computing
and operating on expression sets rather than computing a single derivative for
each expression, we can achieve competitive performance for automata
construction as well as at matching time.

Similarities and differences between our method and existing regular expression
engines supporting submatching have been shown. We have evaluated several
algorithms and shown how our algorithm improves upon them.

Future efforts will be directed at improving usability and performance of the
tool, so that it can be used as a drop-in replacement for \ocamllex{} with
additional functionality, while retaining or improving upon its performance.

Some ideas for future development:

\begin{itemize}

   \item Add Unicode support, building on the ideas implemented in \mlulex{} and
      presented in \cite{re-deriv}. This would improve compile time performance
      even for non-Unicode patterns.

   \item Perform static analysis on regular expressions and the resulting
      automaton to provide better error messages, both at compile time and at
      runtime.

   \item Provide an option to determinise the NFA and minimise the resulting
      DFA, at the expense of increased compile time.

   \item Implement an ML code generator producing mutually recursive functions
      in addition to the current table-based backend.

   \item Investigating the possibilities within a generic submatching based
      lexer engine.

      It would be interesting to include the semantic action functions in the
      AST\footnote{Abstract Syntax Tree} data structure representing patterns.
      These functions would replace the variable names and using
      GADTs\footnote{Generalised Algebraic Data Types}, we might be able to
      construct a statically typed heterogeneous matching environment. Initial
      attempts at this failed, so further research is required.

   \item Make the generic lexer engine immediate, in that it does not defer
      actions but immediately calls the semantic action, discarding its value if
      the parent pattern does not match.

      Further research is required here, as well, to decide whether this method
      is feasible in terms of performance.

\end{itemize}

An OCaml implementation of the algorithms presented in chapter \ref{submatching}
and discussed in chapter \ref{implementation} can be downloaded as free and open
source code under the GNU General Public License version 3 from
\url{https://github.com/pippijn/dreml}.


% vim:tw=80
